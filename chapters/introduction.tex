% What is fANOVA?
At its core the fANOVA decomposition provides a method which allows decomposing integrable functions into a sum of orthogonal components.
fANOVA is a foundational method, useful in interpretability of machine learning models (\cite{hooker2004, molnar2025}), uncertainty quantification of complex systems \cite{rahman2014}, non-parametric statistical modelling (see for example \cite{stone1997}), sensitivity analysis \cite{sobol1993sensitivity}, and many more fields.\par

% What is the problem?

A problem however is the mix of formalizations and definitions around the method, partly due to long history, and different streams of science that have used the method.
This already starts with the name of the method. It has been called decomposition into summands of different order \citep{sobol1993sensitivity}, ANOVA representation \citep{sobol1993sensitivity}, functional ANOVA decomposition \citep{hooker2004}, ANOVA dimensional decomposition \citep{rahman2014} - we will refer to it as fANOVA decomposition.
It continues with how the fANOVA decomposition is formalized in the literature. Every paper uses different notation, a (slightly) different set of assumptions, and interprets fANOVA from a probabilistic point using the expected value or from a more deterministic mathematical viewpoint using the integral.
As we will see, the definitions are equivalent and come together under concept of orthogonal projections. This parallel might be hard to recognize when one start learning about fANOVA.\par

Given this state of affairs, there is a need for a comprehensive overview of fANOVA-related work and a need to clean up various notations and definitions that in the end state the same.
Bringing clarity into the fANOVA landscape is more relevant than ever as the method has attracted attention in recent IML literature (see for example \cite{hu2025}).
It is used to build novel inherently interpretable models, but the theoretical foundation often goes unaddressed.\par

% What is (a possible) solution?
This paper aims to provide an accessible and intuitive introduction to the fANOVA decomposition while remaining mathematically rigorous.
It can be viewed as a handbook of the fANOVA decomposition that will help researchers and practitioners
to understand the mathematical background of this method as well as its more applied aspects.
This work is organized as follows: It starts with historical context and how the method has evolved over time; we then give formal introduction to classical fANOVA as well as the generalization to dependent inputs. We will outline possible estimation schemes, particularly relevant for an application of the method. Next we illustrate characteristics of the classical fANOVA based on analytical examples; before concluding with a discussion of the method's limitations and possible future research directions.
