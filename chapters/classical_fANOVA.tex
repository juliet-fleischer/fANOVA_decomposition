We start by defining the fANOVA decomposition in a very general form (which is independent of distribution assumptions or anything of the sort).

\begin{definition}
Let $y$  denote a mathematical model with input denoted by $X_1, \dots, X_N$. The functional ANOVA (fANOVA) decomposition of $y$ takes the form:
\begin{equation}
    y(\boldsymbol{X}) = \sum_{u \subseteq \{1, \dots, N\}} y_{u}(\boldsymbol{X}_u),
    \label{eq:fanova_decomposition}
\end{equation}
where $u \subseteq \{1, \dots, N\} = \{ \{1\}, \{2\}, \{1, 2\}, \dots, \{1, \dots, N\} \}$ is the set, which contains all subsets of the indices $1, \dots, N$.
\end{definition}


The decomposition consists of $2^N$ terms and gives us a very general expression which's specific form is determined by the assumptions about the input variables and integration measure.

\subsection{Classical fANOVA}
For the classical case, originally proposed by \cite{sobol1993sensitivity}, we make the assumption of independent identically distributed (i.i.d.) input variables. This means we work with the measure space $(\mathbb{R}^n, \mathcal{B}(\mathbb{R}^n), \nu)$, and with a general measure $\nu$ defined on it.
Under independence the joint probability density function (pdf) is given by the product over the marginal pdfs, i.e. \(f_{\boldsymbol{X}}(\boldsymbol{x}) \, d\nu(\boldsymbol{x}) = \prod_{i=1}^{N} f_{X_i}(x_i) \, d\nu(x_i)\), where \(f_{X_i}: \mathbb{R} \rightarrow \mathbb{R}_{0}^{+}\) is the marginal probability density function of \(X_i\) defined on $(\Omega_i, \mathcal{F}_i, \nu_i)$ (or the previously defined measure space?).

Next, we formulate a condition, proposed by \cite{rahman2014}, which we would like to hold for the fANOVA terms to be well-defined and interpretable.\par
\textbf{The strong annihilating conditions} require that the fANOVA terms integrate to zero w.r.t the individual variables contained in $u$ and weighted by the individual marginal pdfs:
\begin{equation}
    \int y_u(\boldsymbol{x}_u) f_{X_i}(x_i) \, d\nu(x_i) = 0, \quad \text{for} \ i \in u \neq \emptyset.
    \label{eq:strong_annihilating_conditions}
\end{equation}

\begin{proposition}
    Given the strong annihilating conditions, the fANOVA components are centered around zero. The constant term is the only exception.
\begin{equation}
    \int y_u(\boldsymbol{x}_u) f_{\boldsymbol{X}}(\boldsymbol{x}) \, d\nu (\boldsymbol{x}) := \mathbb{E}[y_u(\boldsymbol{X}_u)] = 0
    \label{eq:zero_mean_c}
\end{equation}
\end{proposition}
\begin{proof}
\begin{align*}
    \mathbb{E}[y_u(\boldsymbol{X}_u)] &:= = \int_{\mathbb{R}^{N}} y_u(\boldsymbol{x_u}) f_{\boldsymbol{X}}(\boldsymbol{x}) \, d\nu (\boldsymbol{x}) \\
    &= \int_{\mathbb{R}^{|u|}} y_u(\boldsymbol{x_u}) f_{\boldsymbol{X}_u}(\boldsymbol{x}_u) \, d\nu (\boldsymbol{x}_u) \\
    &= \int_{\mathbb{R}^{|u|}} y_u(\boldsymbol{x_u}) \prod_{i \in u} f_{X_i}(x_i) \, d\nu (\boldsymbol{x}_u) \\
    &= \int_{\mathbb{R}^{|u|-1}} \int_{\mathbb{R}} y_u(\boldsymbol{x_u}) f_{X_i}(x_i) \, dx_u \prod_{j \in u, j \neq i} f_{X_j}(x_j) = 0
\end{align*}
\end{proof}

\begin{proposition}
    Given the strong annihilating conditions, it follows that the fANOVA terms are orthogonal to each other. If two sets of indices are not completely equivalent, i.e. $\emptyset \neq u \subseteq \{1, \dots, N\}, \emptyset \neq v \subseteq \{1, \dots, N\}, \text{ and } u \neq v$, then it holds that:
\begin{equation}
    \int y_u(\boldsymbol{x_u}) y_v(\boldsymbol{x_v}) f_{\boldsymbol{X}}(\boldsymbol{x}) d\nu (\boldsymbol{x}) = \mathbb{E}[y_u(\boldsymbol{X}_u) y_v(\boldsymbol{X}_v)] = 0
    \label{eq:orthogonality_c}
\end{equation}
\end{proposition}

\begin{proof}
    \begin{align*}
    \mathbb{E}[y_u(\boldsymbol{X}_u) y_v(\boldsymbol{X}_v)] &= \int_{\mathbb{R}^{\mathbb{N}}} y_u(\boldsymbol{x_u}) y_v(\boldsymbol{x_v}) f_{\boldsymbol{X}}(\boldsymbol{x}) \, d\nu (\boldsymbol{x}) \\
    &= \int_{\mathbb{R}^{\mathbb{N}}} y_u(\boldsymbol{x_u}) y_v(\boldsymbol{x_v}) \prod_{i=1}^{N} f_{X_i}(x_i) \, d\nu (x_i) \\
    &= \int_{\mathbb{R}^{\mathbb{N}-1}} \int_{\mathbb{R}} y_u(\boldsymbol{x_u}) y_v(\boldsymbol{x_v}) f_{X_i}(x_i) \, dx_u \prod_{j \in \{1, \dots, N\}, j \neq i} f_{X_j}(x_j) = 0
\end{align*}
\end{proof}

This means that fANOVA terms are ``fully orthogonal'' to each other, meaning not only terms of different order are orthogonal to each other but also terms of the same order are. Zero-mean and orthogonality are desirable and important properties because they ensure that the fANOVA terms can be interpretate as isolated effects of the specific variable(s). The term $y_1$, for example, captures the isolated main effects of $X_1$; there is no other effect mixed into it, which $X_1$ might have through interactions with other variables. From the lense of interpretability, this distinguishes the fANOVA decomposition from methods such as partial dependence (PD) or Shapley values.\par

% And this is how the components finally look like
\subsubsection*{Construction of the fANOVA Terms}
The individual fANOVA terms for the variables with indices in $u$ are constructed by integrating the original function $y(\boldsymbol{X})$ w.r.t all variables expect for the ones in $u$, and subtracting the lower order terms. Intuitively the integral is averaging the original function over all other variables expect the ones of interest, which makes sense as we are then left with a function of the variables of interest only. Subtracting lower order terms corresponds to account for effects that are already explained by other variables or interactions so that we obtain the isolated effects.\par
Since $u = \emptyset$ for the constant term, we integrate w.r.t all variables:
\begin{equation}
    y_{\emptyset} = \int y(\boldsymbol{x}) \prod_{i=1}^{N} f_{X_i}(x_i) \, d\nu (x_i) = \mathbb{E}[y(\boldsymbol{X})].
    \label{eq:intercept_classical}
\end{equation}
For all other effects $\emptyset \neq u \in \{1, \dots, N\}$ we can calculate:
\begin{equation}
    y_u(\boldsymbol{X}_u) = \int y(\boldsymbol{X}_u, \boldsymbol{x}_{-u}) \prod_{i=1, i \notin u}^{N} f_{X_i}(x_i) \, d\nu (x_i)- \sum_{v \subsetneq u} y_v(\boldsymbol{X}_v).
    \label{eq:fanova_components_classical}
\end{equation}
Notice that this definition relies on a product-type measure rooted in the independence assumption. We will see what changes when we let got of this assumption in the next section.\par
As suggested earlier, the fANOVA components offer a clear interpretation of the model, decomposing it into main effects, two-way interaction effects, and so on. This is why fANOVA decomposition has received increasing attention in the IML and XAI literature, holding the potential for a global model-agnostic explanation method of black box models.\par
% So this is what we did so far: we defined the decomposition, defined the terms it is made up of, and looked a bit deeper at their mathematical properties and how to satisfiy them.
% There is an alternative way to define fANOVA, or just another way of looking at it, which builds on the connection between orthogonal projections and conditional expected values.

\subsubsection{Example: Multivariate Normal Inputs}

Before further investigating the fANOVA decomposition, let us consider the following function as example: \(g = a + X_1 + 2X_2 + X_1 X_2\). We assume that $\boldsymbol{X} = (X_1, X_2)^T$ follows a standard MVN distribution, so the $\boldsymbol{\mu} = (0, 0)^T$ and the covariance matrix $\boldsymbol{\Sigma} = \begin{pmatrix} 1 & 0 \\ 0 & 1 \end{pmatrix}$.

From the properties of the MVN, we know that marginal distributions are standard normal:
\[
X_i \sim \mathcal{N}(0, 1) \quad \text{for } i = 1, 2
\]

We also know that the conditional distributions are given by:
\[
X_1 \mid X_2 = x_2 \sim \mathcal{N}(0, 1), \quad
X_2 \mid X_1 = x_1 \sim \mathcal{N}(0, 1)
\]

\subsubsection*{Case 1: Independent Inputs}
The classical fANOVA decomposition we covered so far assumes $\rho_{12} = 0$. Computing the fANOVA decomposition of $g(x_1, x_2)$ by hand, we start with the constant term and make use of formulation via the expected value:
\[
y_0 = \mathbb{E}[g_{1}(X_1, X_2)] = \mathbb{E}[a + X_1 + 2X_2 + X_1X_2] = \mathbb{E}[a] + \mathbb{E}[X_1] + 2\mathbb{E}[X_2] + \mathbb{E}[X_1X_2]
\]
Making use of the independence assumption of $X_1$ and $X_2$, the last term can be written as the product of the expected values. Additionally, given the zero-mean property, all terms, except for the constant, vanish, and we obtain:
\[
y_0 = \mathbb{E}[a] + \mathbb{E}[X_1] + 2\mathbb{E}[X_2] + \mathbb{E}[X_1]\mathbb{E}[X_2] = a
\]

Under zero-mean constraint and independence, the main effects and the interaction effect can be computed as follows:
\begin{align*}
y_1(x_1) &= \mathbb{E}_{X_2}[g_{1}(x_1, X_2)] - y_0 \\
&= \mathbb{E}_{X_2}[a + x_1 + 2X_2 + x_1X_2] - a \\
&= x_1 + 2\mathbb{E}[X_2] + x_1\mathbb{E}[X_2] = x_1\\
y_2(x_2) &= \mathbb{E}_{X_1}[g_{1}(X_1, x_2)] - y_0 \\
&= \mathbb{E}_{X_1}[a + X_1 + 2x_2 + X_1x_2] - a \\
&= \mathbb{E}_{X_1}[X_1] + 2x_2 + x_2\mathbb{E}_{X_1}[X_1] = 2x_2\\
y_{12}(x_1, x_2) &= \mathbb{E}[g_{1}(x_1, x_2)] - y_0 - y_1(x_1) - y_2(x_2) \\
&= a + x_1 + 2x_2 + x_1x_2 - a - x_1 - 2x_2 = x_1x_2
\end{align*}

It comes as no surprise that in this simple case the fANOVA decomposition does not provide any additional insights, as the isolated effects can be directly seen from the function.
We show this simple example nevertheless to illustrate at which step which assumption is used.
This will make clearer what breaks down when we generalize to dependent variables.

\subsubsection{fANOVA as projection}
In the following we revisit the fANOVA decomposition from the view of orthogonal projections. The section is based on \cite{Vaart_1998}.
Having this perspective on the fANOVA decomposition is useful helps in bridging different notations of the method (e.g. via expected value or via integral) and also supports in understanding the generalization of fANOVA in section~\ref{generalization}.\par

When we define the constant term $y_\emptyset$ our goal is to best approximate the original function $y$ by a constant function. In other words, we want to minimize the squared difference between $y$ and a constant function $g(x) = a$ over all possible constant functions. The solution is the orthogonal projection of $y$ onto the linear subspace of all constant functions $\mathcal{G}_0 = \{g(x) = a; a \in \mathbb{R}\}$. In a probabilistic context, we want to minimize the expected squared different between the random variables $y(\boldsymbol{X})$ and $a$, which turns out to be equivalent to the expected value of the random variable \citep{Vaart_1998}. So intuitively, in the absence of any additional information, the expected value is our best approximation of $y$. More formally we can write:
\begin{align*}
    \Pi_{\mathcal{G}_0}y
    &= \arg \min_{g_0 \in \mathcal{G}_0} \|y - g_0\|^2 \\ % here we still focus on the functions (function space view)
    &= \arg \min_{a_0 \in \mathbb{R}} \mathbb{E}[(y(\boldsymbol{X}) - a)^2] \\ % here we switch to the probabilistic view, focus on RV
    &= \mathbb{E}[y(\boldsymbol{X})] = y_0
\end{align*}
The main effect $y_i(x_i)$ is the projection of $y$ onto the subspace of all functions that only depend on $x_i$, i.e. $\mathcal{G}_i = \{g(x) = g_i(x_i)\}$. There is no need for additional constraints since subtracting lower order terms ensures that orthogonality and zero mean are fulfilled.
The conditional expected value of $\mathbb{E}[y(\boldsymbol{X}) \mid X_i = x_i]$ is the solution to the minimization problem \citep{Vaart_1998}, and the conditional expected value is also a way to express the fANOVA terms \citep{muehlenstaedt2012}:
\begin{align*}
    (\Pi_{\mathcal{G}_i}y)(.) - y_0
    &= \arg \min_{g_i \in \mathcal{G}_i} \|y - g_i\|^2 - y_0\\
    &= \arg \min_{g_i \in \mathcal{G}_i} \mathbb{E}[(y(\boldsymbol{X}) - g_i(X_i))^2] - y_0 \\
    &= \mathbb{E}[y(\boldsymbol{X}) \mid X_i = .] - y_0 = y_i(.)
\end{align*}

The two-way interaction effect $y_{ij}(.,.)$ is the projection of $y$ onto the subspace of all functions that depend on $x_i$ and $x_j$. i.e. $\mathcal{G}_{i,j} = \{g(x) = g_{ij}(x_i, x_j)\}$. Again, we account for lower-order effects by subtracting the constant term and all main effects:
\begin{align*}
    (\Pi_{\mathcal{G}_{ij}}y)(.;.) - (y_0 + y_i(.) + y_j(.))
    &= \arg \min_{g_{ij} \in \mathcal{G}_{ij}} \|y - g_{i, j}\|^2 - (y_0 + y_i(.) + y_j(.))\\
    &= \arg \min_{g_{ij} \in \mathcal{G}_{ij}} \mathbb{E}[(y(\boldsymbol{X}) - g(., .))^2] - (y_0 + y_i(.) + y_j(.))\\
    &= \mathbb{E}[y(\boldsymbol{X}) | X_j = x_j, X_i = x_i] - (y_0 + y_i(.) + y_j(.)) = y_{ij}(.;.)
\end{align*}

In general, we can write for a subset of indices $u \subseteq \{1, \dots, N\}$ and the subspace $\mathcal{G}_u = \{g(\boldsymbol{x}) = g_u(\boldsymbol{x}_u)\}$:
\begin{align*}
    (\Pi_{\mathcal{G}_u}y)(.) - \sum_{v \subsetneq u} y_v(.)
    &= \arg \min_{g_u \in \mathcal{G}_u} \|y - g_u\|^2 - \sum_{v \subsetneq u} y_v(.)\\
    &= \arg \min_{g_u \in \mathcal{G}_{u}} \mathbb{E}[(y(\boldsymbol{X}) - g(.))^2] - \sum_{v \subsetneq u} y_v(.)\\
    &= \mathbb{E}[y(\boldsymbol{X}) | X_{u} = x_u] - \sum_{v \subsetneq u} y_v(x) = y_u(.),
\end{align*}
which means that we project $y$ onto the subspace spanned by the own terms of the fANOVA component to be defined, while accounting for all lower-order terms.

\subsubsection*{Projection of the differences or subtracting from the projection}
Thanks to the equivalence of the conditional expected value and projections we established the mathematical foundation/ mechanism of fANOVA.
Next we want to highlight that instead of subtracting the lower order terms from the projection, it is just as valid to first subtract lower order terms and project $y$ on what is left.
We can find both formulations in the literature.
For example, \cite{muehlenstaedt2012} subtracts from the projection and defines:
\begin{align*}
    y_u(\boldsymbol{x}_u) &:=
    \mathbb{E}[y(\boldsymbol{X}) | \boldsymbol{X}_{u} = \boldsymbol{x}_u] - \sum_{v \subsetneq u} y_v(\boldsymbol{x}) \\
    & \int_{-u} y(\boldsymbol{x}) d \nu(\boldsymbol{x}_{-u}) - \sum_{v \subsetneq u} y_v(\boldsymbol{x})
\end{align*}
\cite{hooker2004} takes the alternative view and defines the fANOVA components via the integral, which can be rewritten as the expected value:
\begin{align*}
    y_u(\boldsymbol{x}_u)
    &:= \int_{-u} (y(\boldsymbol{x}) - \sum_{v \subsetneq u} y_v(\boldsymbol{x})) d \nu(\boldsymbol{x}_{-u}) \\
    & \mathbb{E}[y(\boldsymbol{X}) - \sum_{v \subsetneq u} y_v(\boldsymbol{x}) | \boldsymbol{X}_{u} = \boldsymbol{x}_u ] 
\end{align*}
The first equivalence in each formulation is simply the definition in each original paper, while the second equivalence holds under the assumption of independent inputs.
% \subsubsection*{Notes \& Clarification}
% Situation: $y(\boldsymbol{X}) \in \Omega, \mathcal{G} \subseteq \Omega, g(\boldsymbol{X}) \in \mathcal{G}$.\par
% \cite{Vaart_1998} tells us that the expected value is equivalent to the projection \cite{muehlenstaedt2012} tells us that the fANOVA terms are equivalent to the conditional expected value.\par


% further analysis of the model via fANOVA decomposition --> variance decomposition
\subsubsection*{Second-moment statistics}
No handbook on fANOVA is complete without at least mentioning \textit{Sobol indices}. This requires us to observe the second moment statistics of the decomposition. We already established that $\mathbb{E}[y(\boldsymbol{X})] = y_{\emptyset}$.
We can also compute the variance of $y(\boldsymbol{X})$ via the fANOVA decomposition. The variance is defined as the expected value of the squared difference between the random variable and its expected value:

We write the sum over $u$ for the sum over $\emptyset \neq u \subseteq \{1, \dots, N\}$ and the sum over $u \neq v$ for the sum over $\emptyset \neq u \subseteq \{1, \dots, N\}, \emptyset \neq v \subseteq \{1, \dots, N\}, u \neq v$.
\begin{align*}
    \sigma^2 := \mathbb{E}[(y(\boldsymbol{X}) - \mu])^2]
    &= \mathbb{E}[(y_{\emptyset} + \sum_{u} y_u({\boldsymbol{X}_u}) - y_{\emptyset})^2] \\
    &= \mathbb{E}[(\sum_{u} y_u({\boldsymbol{X}_u}))^2] \\
    &= \mathbb{E}[\sum_{u} y_u^2({\boldsymbol{X}_u})] + 2 \mathbb{E}[\sum_{u \neq v} y_u({\boldsymbol{X}_u})  y_v({\boldsymbol{X}_v})] \\
    & = \sum_{u} \mathbb{E}[y_u^2({\boldsymbol{X}_u})]
\end{align*}

We can verify that the variance decomposition holds for our example:
\begin{align*}
    Var(a + X_1 + 2X_2 + X_1 X_2) &= Var(X_1) + 4Var(X_2) + Var(X_1X_2) + 2Cov(X_1, X_2) \\
    &= 1 + 4 \cdot 1 + 1 \cdot 1 + 2 \cdot 0 = 6 \\
    &= \mathbb{E}[X_1^2] + 4\mathbb{E}[X_2^2] + \mathbb{E}[X_1^2]\mathbb{E}[X_2^2] + 2Cov(X_1, X_2) \\
    &= \mathbb{E}[y_1^2] + \mathbb{E}[y_2^2] + \mathbb{E}[y_{12}^2] \\
\end{align*}
Studying the variance of the decomposition was the main focus in early works on this method (see e.g. \cite{sobol1993sensitivity}).
From the variance decomposition \cite{sobol1993sensitivity} construct the \textit{Sobol indices}, which are well-known in sensitivity analysis. As it is only one application of the fANOVA decomposition, we will not go into depth here, but we should keep in mind that in most works, the presentation of fANOVA is closely linked to the Sobol indices.
% Second variant of variance decomposition formulated via the expected value
% I could first show this and then show in a Lemma that if the expected value of y^2(X) exists (is smaller than \infinity) then the expected value of y_u^2(X_u) also exists, this would essentially be the translation from Sobols L^2 statement to the expected value wouldn't it?








