The formal setup is based on \cite{rahman2014} and \cite{chastaing2012}.\par
Let $\mathbb{N}, \mathbb{N}_0, \mathbb{R}, \text{ and } \mathbb{R}_0^{+}$ denote the sets of positive integer (natural), nonnegative integer, real, and nonnegative real numbers, respectively. Throughout this thesis, we represent the $k$-dimensional Euclidean space by $\mathbb{R}^k$ and the set of all $k \times k$ real-valued matrices by $\mathbb{R}^{k \times k}$.\par
Let $(\Omega, \mathcal{F}, \nu)$ be a measure space, where $\Omega$ is a sample space, $\mathcal{F}$ is a $\sigma-$algebra on $\Omega$ and $\nu: \mathcal{F} \rightarrow [0, 1]$ is a probability measure. $\mathcal{B}^N$ is the Borel $\sigma$-algebra on $\mathbb{R}^N, N \in \mathbb{N}$.
$\boldsymbol{X} = (X_1, \dots, X_N): (\Omega, \mathcal{F}) \rightarrow (\mathbb{R}^N, \mathcal{B}^N)$ denotes an $\mathbb{R}^N$-valued random vector.\par
We assume that the probability distribution of $\boldsymbol{X}$ is continuous and completely defined by the joint probability density function $f_{\boldsymbol{X}}: \mathbb{R}^N \rightarrow \mathbb{R}_{0}^+$. % $f_{\boldsymbol{X}}$ is the probability density function with respect to (w.r.t.) the measure $\nu$. \par

Let $u$ denote a subset of $\{1, \dots, N\}$ with the complementary set  $-u := \{1, \dots, N\} \backslash{} u$ and cardinality $0 \leq |u| \leq N$. We denote strict inclusion of a subset by $\subsetneq$ and $\subseteq$ allows for equality.
$\boldsymbol{X_u} = (X_{i_1}, \dots, X_{i_{|u|}}), u \neq \emptyset, 1 \leq i_1 < \dots < i_{|u|} \leq N$ is a sub-vector of $\boldsymbol{X}$ and $\boldsymbol{X}_{-u} = \boldsymbol{X}_{\{1, \dots, N\} \backslash{} u}$ is the complementary subvector.

The marginal density function of $\boldsymbol{X_u}$ is $f_u(\boldsymbol{x_u}) := \int_{\mathbb{R}^{N-|u|}} f_{\boldsymbol{X}}(\boldsymbol{x})d \nu (\boldsymbol{x}_{-u})$ for a given set $\emptyset \neq u \subseteq \{1, \dots N\}$.\par
Let $y(\boldsymbol{X}) := y(X_1, \dots, X_N)$ be a real-valued, measurable transformation on $(\Omega, \mathcal{F})$, which represents a probabilistic model with random variables as inputs. The Hilbert space of square-integrable functions $y$ with respect to the induced generic measure $f_{\boldsymbol{X}}(\boldsymbol{x})d \nu (\boldsymbol{x})$ supported on $\mathbb{R}^N$ is given by:
\[
\mathcal{L}^2(\Omega, \mathcal{F}, \nu) = \left\{ y: \Omega \rightarrow \mathbb{R} \; \textit{s.t.} \; \mathbb{E}[y^2(\boldsymbol{X})] < \infty \right\}.
\]
% = \left\{ f(x) : \mathbb{R}^{n} \to \mathbb{R}, \; \textit{s.t.} \; \int f^2(x)\, d\nu(x) < \infty \right\}

The inner product defined as:
\[
\langle y, g \rangle = \int_{\mathbb{R}^N} y(\boldsymbol{x}) g(\boldsymbol{x}) \, f_{\boldsymbol{X}}d\nu(\boldsymbol{x}) = \mathbb{E}[y(\boldsymbol{X})g(\boldsymbol{X})], \quad \forall y,g \in \mathcal{L}^2.
\]
The norm, denoted as $\|.\|$, is defined by:
\[
\|y\| = \sqrt{\langle y, y \rangle} = \sqrt{\int_{\mathbb{R}^N} y^2(\boldsymbol{x}) \, d\nu(\boldsymbol{x})} = \mathbb{E}[y^2(\boldsymbol{X})], \quad \forall y \in \mathcal{L}^2.
\]

We start by defining the fANOVA decomposition in a general form, which is independent of distribution assumptions about the input variables or anything of the sort. The decomposition consists of $2^N$ components and its specific form is determined by the assumptions about the input variables and integration measure.

\begin{definition}\label{def:fanova_decomposition}
Let $y$ denote a mathematical model with input vector $\boldsymbol{X} := (X_1, \dots, X_N)$. 
The functional ANOVA (fANOVA) decomposition of $y$ takes the form:
\begin{equation}
    y(\boldsymbol{X}) = \sum_{u \subseteq \{1, \dots, N\}} y_{u}(\boldsymbol{X}_u).
\end{equation}
The functions $y_u$ are referred to as \emph{fANOVA component functions} 
(or simply \emph{components}) throughout this thesis.
\end{definition}



\subsection{Classical fANOVA}
For his original fANOVA decomposition, Sobol only considered function defined on the unit hypercube, but later work shows that it is no problem to work within the measure space $(\mathbb{R}^N, \mathcal{B}^N, \nu)$.
In any case, we assume that the coordinates $X_1, \dots , X_N$ are independent of each other.
Under independence, we work with a product-type probability measure of $\boldsymbol{X}$ given by \(f_{\boldsymbol{X}}(\boldsymbol{x}) \, d\nu(\boldsymbol{x}) = \prod_{i=1}^{N} f_{\{i\}}(x_i) \, d\nu(x_i)\), where \(f_{\{i\}}: \mathbb{R} \rightarrow \mathbb{R}_{0}^{+}\) is the marginal probability density function of \(X_i\) defined on $(\Omega_i, \mathcal{F}_i, \nu_i)$ with a bounded or an unbounded support on $\mathbb{R}$.

Given this setup, we formulate a condition, proposed by \cite{rahman2014}, which we would like to hold for the fANOVA component functions to be well-defined and interpretable.
\begin{condition}[Strong annihilating conditions, \cite{rahman2014}]\label{cond:strong_annihilating_conditions}
    For the classical fANOVA decomposition we require, that all the nonconstant fANOVA component functions $y_u$ integrate to zero w.r.t. the marginal probability density of each random variable in $u$, that is,
\begin{equation}
    \int_{\mathbb{R}} y_u(\boldsymbol{x}_u) f_{\{i\}}(x_i) \, d\nu(x_i) = 0 \quad \text{for} \ i \in u \neq \emptyset.
\end{equation}
\end{condition}

\begin{proposition}[\cite{rahman2014}]\label{prop:zero_mean_classical}
Given that the strong annihilating conditions are satisfied, 
the fANOVA component functions $y_u$, where 
$\emptyset \neq u \subseteq \{1,\dots,N\}$, are centered around zero:
\begin{equation}
    \int_{\mathbb{R}^N} 
        y_u(\boldsymbol{x}_u)\, f_{\boldsymbol{X}}(\boldsymbol{x}) 
        \, d\nu (\boldsymbol{x})
    = \mathbb{E}[y_u(\boldsymbol{X}_u)] 
    = 0.
\end{equation}
\end{proposition}

\begin{proof}
\begin{align*}
    \mathbb{E}[y_u(\boldsymbol{X}_u)] &:= \int_{\mathbb{R}^{N}} y_u(\boldsymbol{x_u}) f_{\boldsymbol{X}}(\boldsymbol{x}) \, d\nu (\boldsymbol{x}) \\[0.5em]
    &= \int_{\mathbb{R}^{|u|}} y_u(\boldsymbol{x_u}) f_{\boldsymbol{X}_u}(\boldsymbol{x}_u) \, d\nu (\boldsymbol{x}_u) \\[0.5em]
    &= \int_{\mathbb{R}^{|u|}} y_u(\boldsymbol{x_u}) \prod_{j \in u} f_{\{j\}}(x_j) \, d\nu (\boldsymbol{x}_u) \\[0.5em]
    &= \int_{\mathbb{R}^{|u|-1}} \int_{\mathbb{R}} y_u(\boldsymbol{x_u}) f_{\{i\}}(x_i) \, d\nu(x_i) \prod_{j \in u, j \neq i} f_{\{j\}}(x_j) \, d\nu (x_{u \setminus \{i\}}) = 0
\end{align*}
\end{proof}

\begin{proposition}[\cite{rahman2014}]\label{prop:orthogonality_classical}
    Given the strong annihilating conditions are satisfied, two distinct fANOVA component functions $y_u$ and $y_v$, where $\emptyset \neq u \subseteq \{1,\ldots,N\}$, $\emptyset \neq v \subseteq \{1,\ldots,N\}$, and $u \neq v$, are orthogonal; i.e., they satisfy
\begin{equation}
    \int_{\mathbb{R}^N} y_u(\boldsymbol{x_u}) y_v(\boldsymbol{x_v}) f_{\boldsymbol{X}}(\boldsymbol{x}) d\nu (\boldsymbol{x}) = \mathbb{E}[y_u(\boldsymbol{X}_u) y_v(\boldsymbol{X}_v)] = 0.
\end{equation}
\end{proposition}

\begin{proof}
Since $u \neq v$, there exists at least one index contained in exactly one of the sets.
Without loss of generality, we pick $i \in u \setminus v$.
Then $y_v(\boldsymbol{x_v})$ is independent of $x_i$, and assuming the strong annihilating conditions hold, we have:
\[
    \int_{\mathbb{R}} y_u(\boldsymbol{x}_u) f_{\{i\}}(x_i)\,d\nu(x_i) = 0
    \quad \text{for all fixed } \boldsymbol{x}_{u\setminus \{i\}}.
\]
Hence,
\begin{align*}
    \mathbb{E}[y_u(\boldsymbol{X}_u) y_v(\boldsymbol{X}_v)] &= \int_{\mathbb{R}^{N}} y_u(\boldsymbol{x_u}) y_v(\boldsymbol{x_v}) f_{\boldsymbol{X}}(\boldsymbol{x}) \, d\nu (\boldsymbol{x}) \\
    &= \int_{\mathbb{R}^{N}} y_u(\boldsymbol{x}_u) y_v(\boldsymbol{x}_v)
       \prod_{j=1}^N f_{\{j\}}(x_j)\, d\nu(\boldsymbol{x}) \\
    &= \int_{\mathbb{R}^{N-1}}
        \left(\int_{\mathbb{R}} y_u(\boldsymbol{x}_u) f_{\{i\}}(x_i)\,d\nu(x_i)\right)
        y_v(\boldsymbol{x}_v)\prod_{j \neq i} f_{\{j\}}(x_j)\,d\nu(\boldsymbol{x}_{-i})\\
    &= 0.
\end{align*}
\end{proof}


As we have seen, the fANOVA components are ``fully orthogonal'' to each other, meaning not only components of different order are orthogonal to each other but also ones of the same order are.
These properties are desirable because they ensure that the 
fANOVA component functions can be interpreted as 
isolated effects of specific variables or their interactions. 
For example, the component function $y_{\{1\}}$ represents the 
isolated main effect of $X_1$; no other contributions 
involving $X_1$ through interactions with other variables are mixed into it. 
Similarly, the component function $y_{\{1,2\}}$ captures only the 
interaction effect between $X_1$ and $X_2$, while the individual effect 
of $X_1$ is already represented by $y_{\{1\}}$ and therefore does not 
merge into $y_{\{1,2\}}$. 
From the perspective of interpretability, this clean separation of effects 
distinguishes the fANOVA decomposition from alternative methods such as 
partial dependence (PD) or Shapley values.

% And this is how the components finally look like
\subsubsection{Construction of the fANOVA Component Functions}
The individual fANOVA component functions associated with the variables 
indexed by $u$ are obtained by integrating the original function 
$y(\boldsymbol{X})$ with respect to all variables except those in $u$ and 
subtracting the corresponding lower-order components. 
Intuitively, the integration averages out the influence of all other variables, 
leaving a function of the variables of interest only. 
Subtracting the lower-order components removes effects already explained by 
other variables or interactions, yielding the isolated effects of the variables in $u$. 

For the classical fANOVA decomposition, these components can be computed as 
follows (see \cite{rahman2014}).
For the constant component, where $u = \emptyset$, we integrate with respect 
to all variables:
\begin{equation}
    y_{\emptyset} 
    = \int_{\mathbb{R}^N} 
        y(\boldsymbol{x}) 
        \prod_{i=1}^{N} f_{\{i\}}(x_i) 
        \, d\nu (x_i) 
    = \mathbb{E}[y(\boldsymbol{X})].
    \label{eq:intercept_classical}
\end{equation}
For all other components, $\emptyset \neq u \subseteq \{1, \dots, N\}$, we obtain:
\begin{equation}
    y_u(\boldsymbol{X}_u) 
    = \int_{\mathbb{R}^{N- |u|}} 
        y(\boldsymbol{X}_u, \boldsymbol{x}_{-u}) 
        \prod_{i=1, i \notin u}^{N} f_{\{i\}}(x_i) 
        \, d\nu (x_i) 
      - \sum_{v \subsetneq u} y_v(\boldsymbol{X}_v).
    \label{eq:fanova_components_classical}
\end{equation}

Notice that this definition relies on a product-type measure rooted in the independence assumption. We will see what changes when we let got of this assumption in the second part of this section.\par
As suggested earlier, the fANOVA component functions offer a clear interpretation of the model, decomposing it into main effects, two-way interaction effects, and so on. This is why fANOVA decomposition has received increasing attention in the IML literature, holding the potential for a global model-agnostic explanation method of black box models.\par
% So this is what we did so far: we defined the decomposition, defined the terms it is made up of, and looked a bit deeper at their mathematical properties and how to satisfiy them.
% There is an alternative way to define fANOVA, or just another way of looking at it, which builds on the connection between orthogonal projections and conditional expected values.
\subsubsection{Example: Independent Multivariate Normal Inputs}

Throughout this thesis, we use the following simple setup as a running example.

\begin{example}[Running Example]\label{ex:running_example}
Consider the bivariate function
\begin{equation}\label{eq:running_example_function}
    h(x_1, x_2) = a + x_1 + 2x_2 + x_1 x_2,
\end{equation}
which includes both main effects and an interaction term.

Assume the input vector
\[
\boldsymbol{X} = (X_1, X_2)^\mathsf{T}
\]
follows a bivariate standard normal distribution
\[
\boldsymbol{X} \sim \mathcal{N}\!\left(
\begin{pmatrix}0 \\ 0\end{pmatrix},
\begin{pmatrix}
1 & \rho \\ 
\rho & 1
\end{pmatrix}
\right),
\]
covering both independent inputs ($\rho = 0$) and correlated inputs ($\rho \neq 0$).

From properties of the multivariate normal distribution, the marginal distributions are
\[
X_1 \sim \mathcal{N}(0,1), \qquad X_2 \sim \mathcal{N}(0,1),
\]
and the conditional distributions are given by
\[
X_1 \mid X_2=x_2 \sim \mathcal{N}(\rho x_2, 1-\rho^2), \quad
X_2 \mid X_1=x_1 \sim \mathcal{N}(\rho x_1, 1-\rho^2).
\]
\end{example}



This example will allow us to compute and compare the classical and generalized fANOVA decompositions for different correlation structures.\par

The classical fANOVA decomposition we covered so far assumes independence, i.e., $\rho = 0$. 
Here, $X_1$ and $X_2$ are independent and standard normal, so the conditional means vanish, and the classical fANOVA decomposition simplifies considerably. 
Computing the constant component via expectation gives:
\begin{align*}
    h_{\emptyset} &= \mathbb{E}[h(X_1, X_2)] \\
    &= \mathbb{E}[a + X_1 + 2X_2 + X_1X_2] \\
    &= a + \mathbb{E}[X_1] + 2\mathbb{E}[X_2] + \mathbb{E}[X_1]\mathbb{E}[X_2] = a.
\end{align*}
Given the zero-mean property and independence, the main components and the interaction components can be computed as follows:
\begin{align*}
h_{\{1\}}(x_1) &= \mathbb{E}_{X_2}[h(x_1, X_2)] - h_{\emptyset} \\
&= \mathbb{E}_{X_2}[a + x_1 + 2X_2 + x_1X_2] - a \\
&= x_1 + 2\mathbb{E}_{X_2}[X_2] + x_1\mathbb{E}_{X_2}[X_2] = x_1,\\
h_{\{2\}}(x_2) &= \mathbb{E}_{X_1}[h(X_1, x_2)] - h_{\emptyset} \\
&= \mathbb{E}_{X_1}[a + X_1 + 2x_2 + X_1x_2] - a \\
&= \mathbb{E}_{X_1}[X_1] + 2x_2 + x_2\mathbb{E}_{X_1}[X_1] = 2x_2,\\
h_{\{1,2\}}(x_1, x_2) &= \mathbb{E}[h(x_1, x_2)] - h_{\emptyset} - h_{\{1\}}(x_1) - h_{\{2\}}(x_2) \\
&= a + x_1 + 2x_2 + x_1x_2 - a - x_1 - 2x_2 = x_1x_2.
\end{align*}

It comes as no surprise that in this simple case the fANOVA decomposition does not provide any additional insights, as the isolated effects can be directly seen from the function.
We show this simple example nevertheless to illustrate at which step which assumption is used.
This will make clearer what breaks down when we generalize to dependent variables.

\subsubsection{Equality to Hoeffding Decomposition}
As we mentioned in \autoref{sec:related_work} the Hoeffding decomposition laid the groundwork for the fANOVA decomposition.
Here, we want to point out that both decompositions yield the same component functions under the assumption of independent and zero-centered inputs. Though we provide no formal proof, we want to illustrate this with our running example (\autoref{ex:running_example}).

\begin{definition}[Hoeffding decomposition, \cite{ilidrissi2025}]\label{def:hoeffding_decomposition}
Let $y$ denote a real-valued function on $\mathbb{R}^N$ with independent inputs $X_1, \dots, X_N$. The Hoeffding decomposition of $y$ takes the form:
\begin{align}
    y(\boldsymbol{X})
=
\sum_{A \subseteq D} 
y_A(\boldsymbol{X}_A),
\qquad
D := \{1,\dots,N\},
\end{align}
where, for each $A \subseteq D$, the component function $y_A$ is defined by
\begin{align}\label{eq:hoeffding_components}
    y_A(\boldsymbol{X}_A)
=
\sum_{B \subseteq A}
(-1)^{|A|-|B|}
\,\mathbb{E}\!\left[
  y(\boldsymbol{X}) 
  \,\middle|\, 
  \boldsymbol{X}_B
\right].
\end{align}
\end{definition}

We can now apply the Hoeffding decomposition to our running example, while we for now write generally $\mu_1 = \mathbb{E}[X_1]$ and $\mu_2 = \mathbb{E}[X_2]$.\par

For $A=\emptyset$ there is only one subset $B=\emptyset$. 
Substituting this into \autoref{eq:hoeffding_components} of \autoref{def:hoeffding_decomposition} we obtain:
\[
h_{\emptyset}
=
(-1)^{0-0}\,
\mathbb{E}[h(\boldsymbol{X})]
=
\mathbb{E}[h(X_1,X_2)].
\]
We compute
\[
\mathbb{E}[h(X_1,X_2)]
= a + \mathbb{E}[X_1] + 2 \mathbb{E}[X_2] 
  + \mathbb{E}[X_1]\mathbb{E}[X_2]
= a + \mu_1 + 2\mu_2 + \mu_1\mu_2.
\]

Next, the subsets of $A=\{1\}$ are $B=\emptyset$ and $B=\{1\}$, so
\[
h_{\{1\}}(X_1)
=
(-1)^{1-0}\,\mathbb{E}[h(\boldsymbol{X})]
+
(-1)^{1-1}\,\mathbb{E}[h(\boldsymbol{X})|X_1]
=
-\mathbb{E}[h] + \mathbb{E}[h(\boldsymbol{X})|X_1].
\]
Since $X_1$ is independent of $X_2$, the conditional expectation is:
\[
\mathbb{E}[h(\boldsymbol{X})|X_1]
= a + X_1 + 2\mu_2 + X_1 \mu_2,
\]
so the final expression is given by:
\[
h_{\{1\}}(X_1)
= - (a+\mu_1+2\mu_2+\mu_1\mu_2) + (a+X_1+2\mu_2+X_1\mu_2)
= (1+\mu_2)(X_1 - \mu_1).
\]

The subsets of $A=\{2\}$ are $B=\emptyset$ and $B=\{2\}$, so
\[
h_{\{2\}}(X_2)
=
(-1)^{1-0}\,\mathbb{E}[h(\boldsymbol{X})]
+
(-1)^{1-1}\,\mathbb{E}[h(\boldsymbol{X})|X_2]
=
-\mathbb{E}[h(\boldsymbol{X})] + \mathbb{E}[h(\boldsymbol{X})|X_2].
\]
Under independence the conditional expectation is:
\[
\mathbb{E}[h(\boldsymbol{X})|X_2]
= a + \mu_1 + 2X_2 + \mu_1 X_2,
\]
which yields the final expression:
\[
h_{\{2\}}(X_2)
= - (a+\mu_1+2\mu_2+\mu_1\mu_2) + (a+\mu_1+2X_2+\mu_1 X_2)
= (2+\mu_1)(X_2 - \mu_2).
\]

Finally, the subsets of $A=\{1,2\}$ are 
$B=\emptyset$, $B=\{1\}$, $B=\{2\}$, $B=\{1,2\}$. 
We obtain:
\begin{align*}
h_{\{1,2\}}(X_1,X_2)
&= (-1)^{2-0}\mathbb{E}[h(\boldsymbol{X})]
   + (-1)^{2-1}\mathbb{E}[h(\boldsymbol{X})|X_1] \notag\\
&\quad + (-1)^{2-1}\mathbb{E}[h(\boldsymbol{X})|X_2]
   + (-1)^{2-2}\mathbb{E}[h(\boldsymbol{X})|X_1,X_2] \notag\\[0.3em]
&= \mathbb{E}[h(\boldsymbol{X})]
   - \mathbb{E}[h(\boldsymbol{X})|X_1]
   - \mathbb{E}[h(\boldsymbol{X})|X_2]
   + \mathbb{E}[h(\boldsymbol{X})|X_1,X_2].
\end{align*}

We already know:
\[
\mathbb{E}[h(\boldsymbol{X})|X_1,X_2] = h(X_1,X_2) = a + X_1 + 2X_2 + X_1 X_2.
\]
Thus:
\begin{align}
h_{\{1,2\}}(X_1,X_2)
&= (a+\mu_1+2\mu_2+\mu_1\mu_2)
   - (a+X_1+2\mu_2+\mu_2 X_1) \notag\\
&\quad - (a+\mu_1+2X_2+\mu_1 X_2)
   + (a+X_1+2X_2+X_1 X_2) \notag\\[0.3em]
&= X_1 X_2 - \mu_2 X_1 - \mu_1 X_2 + \mu_1 \mu_2 \notag\\
&= (X_1 - \mu_1)(X_2 - \mu_2).
\end{align}


Bringing everything together, the Hoeffding decomposition of $h(X_1,X_2)$ with general means 
$\mu_1=\mathbb{E}[X_1]$ and $\mu_2=\mathbb{E}[X_2]$ is:
\[
h(X_1,X_2)
=
h_{\emptyset} + h_1(X_1) + h_2(X_2) + h_{12}(X_1,X_2),
\]
with
\[
\begin{aligned}
h_{\emptyset} &= a + \mu_1 + 2\mu_2 + \mu_1\mu_2, \\[0.3em]
h_1(X_1) &= (1+\mu_2)(X_1 - \mu_1), \\[0.3em]
h_2(X_2) &= (2+\mu_1)(X_2 - \mu_2), \\[0.3em]
h_{\{1,2\}}(X_1,X_2) &= (X_1 - \mu_1)(X_2 - \mu_2).
\end{aligned}
\]

Under the special case of centered input variables, as we assumed in the running example, the decomposition simplifies to:
\[
h_{\emptyset}=a,\quad
h_{\{1\}}(X_1)=x_1,\quad
h_{\{2\}}(X_2)=2x_2,\quad
h_{\{1,2\}}(X_1,X_2)=x_1 x_2,
\]
which coincides with the fANOVA component functions calculated for the polynomial from our running example (\autoref{eq:running_example_function}).\par
The principle of the Hoeffding decomposition is the same as that of the fANOVA decomposition, but the latter is expressed in a recursive form, making explicit that each component accounts for the contributions of lower-order components.
In addition, the fANOVA component functions are themselves are zero-centered by construction.


\subsubsection{fANOVA via Projection}
In the following we revisit the fANOVA decomposition from the view of orthogonal projections.
For this section the parallel between the (conditional) expected value and orthogonal projections formulated in \cite{Vaart_1998} is crucial.
Having this perspective on the fANOVA decomposition helps in bridging different notations of the method (e.g. via expected value or via integral) and also supports in understanding the generalization of fANOVA later in this section. First we define generally what an orthogonal projection is, and then we will use the idea in the context of fANOVA.\par

\begin{definition}[Orthogonal Projection, {adapted from \cite{nagler2024mathstat}}]\label{def:orthogonal_projection}
    Let $\mathcal{G} \subset \mathcal{L}^2$ denote a linear subspace. The projection of $y$ onto $\mathcal{G}$ is defined by the function $\Pi_{\mathcal{G}}y$ which minimizes the distance to $y$ in $\mathcal{L}^2$:
\begin{align}
    \Pi_{\mathcal{G}}y = \arg\min_{g \in \mathcal{G}} \|y - g\|^2
= \arg\min_{g \in \mathcal{G}} \mathbb{E}[(y(\boldsymbol{X}) - g(\boldsymbol{X}))^2].
\end{align}
\end{definition}

When we define the constant component $y_\emptyset$ our goal is to best approximate the original function $y$ by a constant function. In other words, we want to minimize the squared difference between $y$ and a constant function $g_0(x) = a$ over all possible constant functions. The solution is the orthogonal projection as given by \autoref{def:orthogonal_projection} of $y$ onto the linear subspace of all constant functions $\mathcal{G}_0 = \{g(x) = a; a \in \mathbb{R}\}$.\par
In a probabilistic context, we want to minimize the expected squared different between the random variables $y(\boldsymbol{X})$ and $a$, which turns out to be equivalent to the expected value of the random variable \citep{Vaart_1998}. So intuitively, in the absence of any additional information, the expected value is our best approximation of $y$. More formally we can write for the constant component $y_{\emptyset}$:
\begin{align*}
    \Pi_{\mathcal{G}_0}y
    &= \arg \min_{g_0 \in \mathcal{G}_0} \|y - g_0\|^2 \\ % here we still focus on the functions (function space view)
    &= \arg \min_{a_0 \in \mathbb{R}} \mathbb{E}[(y(\boldsymbol{X}) - a)^2] \\ % here we switch to the probabilistic view, focus on RV
    &= \mathbb{E}[y(\boldsymbol{X})] = y_{\emptyset}
\end{align*}
The main component $y_{\{i\}}(x_i)$ is the projection of $y$ onto the subspace of all functions that only depend on $x_i$, i.e. $\mathcal{G}_i = \{g(x) = g_{\{i\}}(x_i)\}$. There is no need for additional constraints since subtracting lower order components ensures that we obtain mutual orthogonal, zero-mean components.
The conditional expected value of $\mathbb{E}[y(\boldsymbol{X}) \mid X_i = x_i]$ is the solution to the minimization problem \citep{Vaart_1998}, and the conditional expected value is also a way to express the fANOVA component functions \citep{muehlenstaedt2012}. Thus, for the main component $y_{\{i\}}(x_i)$  we can write:
\begin{align*}
    (\Pi_{\mathcal{G}_i}y)(.) - y_{\emptyset}
    &= \arg \min_{g_i \in \mathcal{G}_i} \|y - g_{\{i\}}\|^2 - y_{\emptyset}\\
    &= \arg \min_{g_i \in \mathcal{G}_i} \mathbb{E}[(y(\boldsymbol{X}) - g_{\{i\}}(X_i))^2] - y_{\emptyset} \\
    &= \mathbb{E}[y(\boldsymbol{X}) \mid X_i = .] - y_{\emptyset} = y_{\{i\}}(.)
\end{align*}

The second-order interaction component $y_{\{i,j\}}(.,.)$ is the projection of $y$ onto the subspace of all functions that depend on $x_i$ and $x_j$. i.e. $\mathcal{G}_{i,j} = \{g(x) = g_{\{i,j\}}(x_i, x_j)\}$. Again, we account for effects captured by lower-order components by subtracting the constan and all main components:
\begin{align*}
    &(\Pi_{\mathcal{G}_{ij}}y)(.;.) - (y_{\emptyset} + y_{\{i\}}(.) + y_{\{j\}}(.)) \\
    &= \arg \min_{g_{\{i,j\}} \in \mathcal{G}_{ij}} \|y - g_{\{i,j\}}\|^2 - (y_{\emptyset} + y_{\{i\}}(.) + y_{\{j\}}(.))\\
    &= \arg \min_{g_{\{i,j\}} \in \mathcal{G}_{ij}} \mathbb{E}[(y(\boldsymbol{X}) - g_{\{i,j\}}(., .))^2] - (y_{\emptyset} + y_{\{i\}}(.) + y_{\{j\}}(.))\\
    &= \mathbb{E}[y(\boldsymbol{X}) | X_j = ., X_i = .] - (y_{\emptyset} + y_{\{i\}}(.) + y_{\{j\}}(.)) = y_{\{i,j\}}(.;.)
\end{align*}

In general, we can write for a subset of indices $u \subseteq \{1, \dots, N\}$ and the subspace $\mathcal{G}_u = \{g(\boldsymbol{x}) = g_u(\boldsymbol{x}_u)\}$:
\begin{align*}
    (\Pi_{\mathcal{G}_u}y)(.) - \sum_{v \subsetneq u} y_v(.)
    &= \arg \min_{g_u \in \mathcal{G}_u} \|y - g_u\|^2 - \sum_{v \subsetneq u} y_v(.)\\
    &= \arg \min_{g_u \in \mathcal{G}_{u}} \mathbb{E}[(y(\boldsymbol{X}) - g_u(.))^2] - \sum_{v \subsetneq u} y_v(.)\\
    &= \mathbb{E}[y(\boldsymbol{X}) | X_{u} = .] - \sum_{v \subsetneq u} y_v(x) = y_u(.),
\end{align*}
which means that we project $y$ onto the subspace $\mathcal{G}_u$ of functions depending only on $X_u$, while subtracting all lower-order components to isolate the effect associated exclusively with $X_u$.\par

On this note, we want to highlight that instead of subtracting the lower order components from the projection, it is just as valid to first subtract lower-order components and project $y$ on what is left.
We can find both formulations in the literature.
For example, \cite{muehlenstaedt2012} subtracts from the projection and defines:
\begin{align*}
    y_u(\boldsymbol{x}_u) &:=
    \mathbb{E}[y(\boldsymbol{X}) | \boldsymbol{X}_{u} = \boldsymbol{x}_u] - \sum_{v \subsetneq u} y_v(\boldsymbol{x}) \\
    &= \int_{\mathbb{R}^{N-|u|}} y(\boldsymbol{x}) d \nu(\boldsymbol{x}_{-u}) - \sum_{v \subsetneq u} y_v(\boldsymbol{x}).
\end{align*}
\cite{hooker2004} takes the alternative view and defines the fANOVA components via the integral, which can be rewritten as the expected value:
\begin{align*}
    y_u(\boldsymbol{x}_u)
    &:= \int_{\mathbb{R}^{N-|u|}} (y(\boldsymbol{x}) - \sum_{v \subsetneq u} y_v(\boldsymbol{x})) d \nu(\boldsymbol{x}_{-u}) \\
    &= \mathbb{E}[y(\boldsymbol{X}) - \sum_{v \subsetneq u} y_v(\boldsymbol{x}) | \boldsymbol{X}_{u} = \boldsymbol{x}_u ].
\end{align*}
The first equivalence in each formulation is simply the definition in each original paper, while the second equivalence holds under the assumption of independent inputs.
% \subsubsection*{Notes \& Clarification}
% Situation: $y(\boldsymbol{X}) \in \Omega, \mathcal{G} \subseteq \Omega, g(\boldsymbol{X}) \in \mathcal{G}$.\par
% \cite{Vaart_1998} tells us that the expected value is equivalent to the projection \cite{muehlenstaedt2012} tells us that the fANOVA terms are equivalent to the conditional expected value.\par


% further analysis of the model via fANOVA decomposition --> variance decomposition
\subsubsection{Variance Decomposition}
Studying the second moments of  a function through the lens of the fANOVA decomposition can be useful, especially with regard to the construction of Sobol indices. This requires us to observe the second moment statistics of the decomposition. We already established that:
$$ \mu := \mathbb{E}[y(\boldsymbol{X})] = y_{\emptyset}.$$

We can also compute the variance of $y(\boldsymbol{X})$ via the fANOVA decomposition.
We write the sum over $u$ for the sum over $\emptyset \neq u \subseteq \{1, \dots, N\}$ and the sum over $u \neq v$ for the sum over $\emptyset \neq u \subseteq \{1, \dots, N\}, \emptyset \neq v \subseteq \{1, \dots, N\}, u \neq v$.
Following the calculations in \cite{rahman2014}, the variance of $y$ is then given by:
\begin{align}\label{eq:variance_decomposition_classical}
\sigma^2 
&:= \mathbb{E}\left[ \left( y(\boldsymbol{X}) - \mu_G \right)^2 \right] \notag \\
&= \mathbb{E} \left[ \left( y_{\emptyset,G} + \sum_{u} y_{u,G}(\boldsymbol{X}_u) - y_{\emptyset,G} \right)^2 \right] \notag \\
&= \mathbb{E} \left[ \left( \sum_{u} y_{u,G}(\boldsymbol{X}_u) \right)^2 \right] \notag \\
&= \sum_{u} \mathbb{E} \left[ y_{u,G}^2(\boldsymbol{X}_u) \right]
 + 2 \mathbb{E}\!\left[\sum_{u \neq v} y_u({\boldsymbol{X}_u}) y_v({\boldsymbol{X}_v})\right] \notag \\ 
& = \sum_{u} \mathbb{E}[y_u^2({\boldsymbol{X}_u})].
\end{align}


where all the cross-terms vanish due to the orthogonality of the fANOVA component functions, i.e. $\mathbb{E}[y_u({\boldsymbol{X}_u})  y_v({\boldsymbol{X}_v})] = 0$ for $u \neq v$.
This means that the variance of $y(\boldsymbol{X})$ can be decomposed into the sum of the variances of the fANOVA component functions.\par

We can verify that the variance decomposition holds for our running example. This means, we verify that the following expression is true:
\[
  \operatorname{Var}\big(h(X_1,X_2)\big)
  = \mathbb{E}\big[h_{\{1\}}^2(X_1)\big]
  + \mathbb{E}\big[h_{\{2\}}^2(X_2)\big]
  + \mathbb{E}\big[h_{\{1,2\}}^2(X_1,X_2)\big]
\]
for the function
\[
  h(x_1,x_2) = a + x_1 + 2x_2 + x_1 x_2 ,
\]
where $X_1,X_2$ are independent with zero-mean and unit variance.

We start with the left-hand side and compute the variance of $h(X_1,X_2)$.
\begin{align*}
\operatorname{Var}\big(h(X_1,X_2)\big)
&= \operatorname{Var}(a + X_1 + 2X_2 + X_1X_2) \\
&= \operatorname{Var}(a) + \operatorname{Var}(X_1) + 4\operatorname{Var}(X_2)
   + \underbrace{\operatorname{Var}(X_1X_2)}_{(\star)}
   + 2\,\operatorname{Cov}(X_1,2X_2) \\
&= 0 + 1 + 4 \cdot 1 + 1 \cdot 1 + 2\cdot 0 = 6, \\[0.3em]
(\star)\;\operatorname{Var}(X_1X_2)
&= \mathbb{E}[X_1^2 X_2^2] - \big(\mathbb{E}[X_1 X_2]\big)^2 \\
&=\mathbb{E}[X_1^2]\,\mathbb{E}[X_2^2]
   - \big(\mathbb{E}[X_1]\,\mathbb{E}[X_2]\big)^2 \\
&= (\operatorname{Var}(X_1)+\mathbb{E}[X_1]^2)\,
   (\operatorname{Var}(X_2)+\mathbb{E}[X_2]^2) - 0 \\
&= 1\cdot 1 = 1.
\end{align*}
For $(\star)$ we used the fact that independence of any measurable map of $X_1$ and $X_2$ follows from independence of $X_1$ and $X_2$ for the second line, and the definition of the variance for the third line.

Next, we go in the other direction and start from the sum of the variances of the fANOVA component functions.
Their variances are given by:
\begin{align*}
\mathbb{E}[h_{\{1\}}^2(X_1)] 
    &= \mathbb{E}[X_1^2] 
    = 1, \\[0.5em]
\mathbb{E}[h_{\{2\}}^2(X_2)] 
    &= \mathbb{E}[(2X_2)^2] 
    = 4, \\[0.5em]
\mathbb{E}[h_{\{1,2\}}^2(X_1,X_2)] 
    &= \mathbb{E}[(X_1 X_2)^2] 
    = 1.
\end{align*}

Therefore, summing everything up we obtain:
\[
 \mathbb{E}[h_{\{1\}}^2(X_1)] + \mathbb{E}[h_{\{2\}}^2(X_2)] + \mathbb{E}[h_{\{1,2\}}^2(X_1,X_2)] 
 = 1 + 4 + 1 = 6
 = \operatorname{Var}\big(h(X_1,X_2)\big).
\]








