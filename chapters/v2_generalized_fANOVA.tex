\subsection{Motivating Example}

\subsection{Formal Introduction to Generalized fANOVA}
In practice assuming independent input variables is often not realistic. When this assumption is violated, it can lead to misleading results \citep{hooker2007}. The need for a generalization to dependent variables is evident. We base this chapter mainly on the generalization of \cite{rahman2014}, while there exists other work from \cite{hooker2007} or \cite{chastaing2012}.\par

Letting go of the independence assumption means that we no longer work with a product-type probability measure. $f_{\boldsymbol{X}}: \mathbb{R} \rightarrow \mathbb{R}_{0}^{+}$ denotes an arbitrary probability density function and $f_{\boldsymbol{X}_u}: \mathbb{R}^u \rightarrow \mathbb{R}_{0}^{+}$ the marginal probability density function of the subset of variables $u \subseteq d$.\par

The definition of the fANOVA decomposition, and the two main properties - zero mean and orthogonality - can be stated in the same way as for the classical case, with s slight change for the orthogonality. We build now on the weak annihilating conditions instead of the strong ones, which results in hierarchical orthogonality instead of full orthogonality.\par

\begin{definition}
    \textbf{Generalized fANOVA decomposition.}
\end{definition}

\textbf{Weak annihilating conditions.}

From this we derive

\begin{proposition}
    % zero mean condition
\end{proposition}

\begin{proposition}
    % hierarchical orthogonality
\end{proposition}

