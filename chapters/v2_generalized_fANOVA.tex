\subsection{Motivating Example}
Recall our example setup of standard MVN input variables and \(g = a + X_1 + 2X_2 + X_1 X_2\) from the previous section~\ref{classical_fANOVA}.
For classical fANOVA we make the assumption of independent inputs, which is often violated in practice. Let us therefore investigate what happens, when we allow for dependency between variables.

\subsubsection*{Case 2: Dependent Inputs}
Now $\rho_{12} \neq 0$, while keeping everything else the same. When we follow the exact same logic as above we obtain the following terms:
\begin{align*}
\tilde{y}_0 &= \mathbb{E}[g(X_1, X_2)] 
= a + \mathbb{E}[X_1] + 2\mathbb{E}[X_2] + \mathbb{E}[X_1 X_2] \\
&= a + \mathbb{E}[X_1 X_2] 
= a + \left( \text{Cov}(X_1, X_2) + \mathbb{E}[X_1]\mathbb{E}[X_2] \right) \\
&= a + \rho_{12} \\
\tilde{y}_1(x_1) 
&= \mathbb{E}[g(X_1, X_2) | X_1 = x_1] - \tilde{y}_0 \\
&= \mathbb{E}[a + x_1 + 2X_2 + x_1 X_2 | X_1 = x_1] - (a + \rho_{12}) \\
&= a + x_1 + 2\mathbb{E}[X_2 | X_1 = x_1] + x_1 \mathbb{E}[X_2 | X_1 = x_1] - a - \rho_{12} \\
&= x_1 + \rho_{12}(2x_1 + x_1^2 - 1) \\
\tilde{y}_2(x_2) 
&= \mathbb{E}[g(X_1, X_2) \mid X_2 = x_2] - \tilde{y}_0 \\
&= \mathbb{E}[a + X_1 + 2x_2 + X_1 x_2 \mid X_2 = x_2] - (a + \rho_{12}) \\
&= a + 2x_2 + x_2 \mathbb{E}[X_1 \mid X_2 = x_2] - a - \rho_{12} \\
&= 2x_2 + \rho_{12}(x_2 + x_2^2 - 1) \\
\tilde{y}_{12}(x_1, x_2) 
&= g(x_1, x_2) - \tilde{y}_0 - \tilde{y}_1(x_1) - \tilde{y}_2(x_2) \\
&= a + x_1 + 2x_2 + x_1 x_2 - (a + \rho_{12}) \\
&- (x_1 + \rho_{12}(2x_1 + x_1^2 - 1)) - (2x_2 + \rho_{12}(x_2 + x_2^2 - 1)) \\
% &= a + x_1 + 2x_2 + x_1 x_2 - (a + \rho_{12}) - (x_1 + 2\rho_{12} x_1 + \rho_{12} x_1^2 - \rho_{12}) - (2x_2 + \rho_{12} x_2 + \rho_{12} x_2^2 - \rho_{12}) \\
&= x_1 x_2 - 2\rho_{12} x_1 - \rho_{12} x_2  - \rho_{12} x_1^2  - \rho_{12} x_2^2 + \rho_{12}
\end{align*}

The fANOVA components are characterized by two central properties zero mean and orthogonality which follow from \autoref{eq:strong_annihilating_conditions}.
When we check if the components $\tilde{y}_0, \tilde{y}_1, \tilde{y}_2, \tilde{y}_{12}$ satisfy these properties, we find out that all components are zero-centred, but not all are orthogonal to each other. We can, for example, immediately see that checking orthogonality between $\tilde{y}_{1}, \tilde{y}_{1,2}$ will yield the expectation over the constant term $\rho_{1,2}$ exactly once, meaning even if all the other expectations cancel out, this constant will remain and the entire expression will be unequal to zero:
\begin{align*}
    \mathbb{E}(\tilde{y}_1(X_1)\tilde{y}_{1,2}(X_1, X_2)) 
    &= \mathbb{E}[(X_1 + 2\rho_{12}X_1 + \rho_{12}X_1^2 - \rho_{12}) \\
    &\quad \cdot (X_1 X_2 - 2\rho_{12} X_1 - \rho_{12} X_2 - \rho_{12} X_1^2 - \rho_{12} X_2^2 + \rho_{12})] \\
    &= \mathbb{E}[X_{1}^2X_2] \ldots - \mathbb{E}[\rho_{12}^2] \neq 0.
\end{align*}

When we no longer have independent inputs naively computing the ``fANOVA decomposition'' does not yield the fANOVA components as it turns out. What we performed in this example is not the fANOVA decomposition for dependent variables. It is Hoeffding decomposition \citep{hoeffding1948} and results in zero mean but not mutually orthogonal component functions. This shows the need for a more involved approach for generalizing fANOVA.
We basically can see from this example that correlation between features distorts the fANOVA component function, it is not pure anymore but this is the whole point of the fANOVA decomposition and central so that it adds sth. IML.

\subsection{Formal Introduction to Generalized fANOVA}
We base this chapter mainly on the generalization of \cite{rahman2014}, while there exists other work from \cite{hooker2007} or \cite{chastaing2012}.\par

Letting go of the independence assumption means that we no longer work with a product-type probability measure. $f_{\boldsymbol{X}}: \mathbb{R} \rightarrow \mathbb{R}_{0}^{+}$ denotes an arbitrary probability density function and $f_{\boldsymbol{X}_u}: \mathbb{R}^u \rightarrow \mathbb{R}_{0}^{+}$ the marginal probability density function of the subset of variables $u \subseteq d$. Classical fANOVA boils down to integration w.r.t. the uniform measure and in generalized fANOVA we integrate w.r.t. the distribution of $(X_1, \dots, X_n)$.\par

% How is generalized fANOVA denoted? (bc the definition perse doesn't change, we still write the model as a sum of basis components of increasing dimensionality)
\begin{definition}
    \textbf{Generalized fANOVA decomposition.}
    We denote the generalized functional fANOVA decomposition as:
    \begin{equation}
        y(\boldsymbol{X}) = \sum_{u \subseteq \{1, \dots, N\}} y_{u, G}(\boldsymbol{X}_u)
    \end{equation}
\end{definition}
The subscript $G$ indicates that we are working with the generalized fANOVA components.
The main question is, how one can build fANOVA components that still satisfy the desired properties of zero mean and orthogonality under dependent inputs.

% But how do the components look like now? This is a difference
\subsubsection*{Construction of the Generalized fANOVA Terms}
While the constant term requires no change in definition, the motivating example in the beginning of this section showed that the non-constant terms need some additional terms to ensure orthogonality. \cite{rahman2014} defines the generalized components as follows:
\begin{align}
y_{\emptyset, G} &= \int_{\mathbb{R}^N} y(\boldsymbol{x}) f_{\boldsymbol{X}}(\boldsymbol{x}) \, d\boldsymbol{x}, \tag{4.5a} \\
y_{u,G}(\boldsymbol{X}_u) &= \int_{\mathbb{R}^{N - |u|}} y(\boldsymbol{X}_u, \boldsymbol{x}_{-u}) f_{\boldsymbol{X}_{-u}}(\boldsymbol{x}_{-u}) \, d\boldsymbol{x}_{-u}
- \sum_{v \subset u} y_{v,G}(\boldsymbol{X}_v) \notag \\
&\quad - \sum_{\substack{\emptyset \neq v \subseteq \{1, \ldots, N\} \\ v \cap u \neq \emptyset, v \not\subset u}}
\int_{\mathbb{R}^{|v \cap -u|}} y_{v,G}(\boldsymbol{X}_{v \cap u}, \boldsymbol{x}_{v \cap -u}) 
f_{v \cap -u}(\boldsymbol{x}_{v \cap -u}) \, d\boldsymbol{x}_{v \cap -u}. \tag{4.5b}
\label{eq:generalized_fanova_components_rahman}
\end{align}
The first part of the non-constant components looks very similar to the classical formulation, but instead of the product pdf we use the joint pdf of all variables except for the ones of interest. The terms we subtract include not only lower order fANOVA terms but also (not yet computed) higher order fANOVA terms, which depend on the variable of interest.
This means, we account for all the terms in which the term of interest is somehow involved in. This is necessary to ensure a form of orthogonality under dependent inputs but also means that solving the terms sequentially, as in the classical case and our naive approach, is not working anymore.

% What conditions do they fulfill? Does it differ? Yes, slightly.
If components are constructed in this way, we can ensure that they have zero mean and satisfy a milder from of orthogonality - hierarchical orthogonality, which means that components of different order are orthogonal to each other while components of the same order are not.
\begin{proposition}
    % zero mean condition
    The generalized fANOVA components $y_{u, G}$, with $\emptyset \neq u \subseteq \{1, \ldots, N\}$, are centred around zero:
\begin{equation}
    \mathbb{E}[y_{u, G}(\boldsymbol{X}_u)] := \int y_{u, G}(\boldsymbol{x}_u) f_{\boldsymbol{X}}(\boldsymbol{x}) \, d\nu (\boldsymbol{x}) = 0
    \label{eq:zero_mean_g}
\end{equation}
\end{proposition}

\begin{proposition}
    % hierarchical orthogonality
    The fANOVA components are hierarchically orthogonal. This means that for two components $y_{u, G}$ and $y_{v, G}$ with $u \subset v, \emptyset \neq u \subseteq \{1, \ldots, N\}, \emptyset \neq v \subseteq \{1, \ldots, N\} $ it holds that:
\begin{equation}
    \mathbb{E}[y_{u, G}(\boldsymbol{X}_u)y_{v, G}(\boldsymbol{X}_v)] := \int y_{u, G}(\boldsymbol{x}_u) y_{v, G}(\boldsymbol{X}_v) f_{\boldsymbol{X}}(\boldsymbol{x}) \, d\nu (\boldsymbol{x}) = 0
\end{equation}
\label{eq:orthogonality_g}
\end{proposition}

% So the constraint we set to the optimization problem is modified, milder.
To ensure that these statements hold for the generalized fANOVA components, we need to set the weak annihilating conditions.
They fulfill the same function as the strong annihilating conditions do in the classical case but work with the joint density of the variables of interest, instead of the individual marginal probability density functions.
{\color{red}This makes sense, because when there are dependencies between variables then the individual pdfs would not assign the correct weight to each function value as they ignore the dependence between features in $u$.}
\begin{proposition}
    \textbf{Weak annihilating conditions.}
    To ensure the two desired properties of the generalized fANOVA components (zero mean, hierarchical orthogonality), we require the weak annihilating conditions:
\begin{equation}
    \int_{\mathbb{R}} y_{u, G}(\boldsymbol{x}_u) f_{\boldsymbol{X}_u}(\boldsymbol{x}_u) d\nu (x_i) = 0 \quad \text{for} \quad i \in u \neq \emptyset
\end{equation}
\end{proposition}

To show that zero mean and hierarchical orthogonality follow from the weak annihilating conditions, \cite{rahman2014} makes use of Fubini's theorem (see his proof in section 4 of his paper - or should I write the commented proof here?).\par

% Alternative definition of components by Hooker
\cite{hooker2007} offers an alternative definition of the generalized fANOVA components\footnote{We modified the notation from the original work to match the notation of \cite{rahman2014} and the rest of this thesis.}:
\begin{equation}
\left\{ y_{u, G}(x_u) \,\middle|\, u \subseteq d \right\}
= \arg\min_{\{g_u \in L^2(\mathbb{R}^u)\}_{u \subseteq d}} 
\int \left( \sum_{u \subseteq d} g_u(x_u) - y(x) \right)^2 f_{\boldsymbol{X}}(\boldsymbol{x}) \, d \nu (\boldsymbol{x})
\label{eq:generalized_fanova_components_hooker}
\end{equation}
In Hookers definition we recognize a projection problem. We are simultaneously finding the set of components functions $g_u$ that minimize the weighted squared difference to the original function $y$ (under zero mean and hierarchical orthogonality constraint), which is exactly the definition of a projection of $y$ onto a specific subspace $\mathcal{G}$, which we defined generally in section ~\ref{background}.\par

A crucial difference to the classical case is that both versions of the generalized components are defined in dependence of each other (\autoref{eq:generalized_fanova_components_rahman}, \autoref{eq:generalized_fanova_components_hooker}).

Let us come back to our example from the beginning. The goal is to write
\[
g(x_1, x_2) = y_{\emptyset, G} + y_{\{1\}, G}(x_1) + y_{\{2\}, G}(x_2) + y_{\{1,2\}, G}(x_1, x_2)
\]
under dependent inputs. We present two ways in which the problem solution can be stated.\par

\subsubsection*{Rahman method}
The system to find the generalized fANOVA components for $g$ according to \cite{rahman2014} method looks as follows:
\begin{align}
    y_{\emptyset, G} &= \int_{\mathbb{R}^2} g(x_1, x_2)\, f(x_1, x_2)\, dx_1 dx_2 \label{eq:y0} \\
    y_{\{1\}, G}(x_1) &= \int_{\mathbb{R}} g(x_1, x_2)\, f_2(x_2)\, dx_2
    - y_{\emptyset, G}
    - \int_{\mathbb{R}} y_{\{1,2\}, G}(x_1, x_2)\, f_2(x_2)\, dx_2 \label{eq:y1} \\
    y_{\{2\}, G}(x_2) &= \int_{\mathbb{R}} g(x_1, x_2)\, f_1(x_1)\, dx_1
    - y_{\emptyset, G}
    - \int_{\mathbb{R}} y_{\{1,2\}, G}(x_1, x_2)\, f_1(x_1)\, dx_1 \label{eq:y2} \\
    y_{\{1,2\}, G}(x_1, x_2) &= g(x_1, x_2) - y_{\emptyset, G} - y_{\{1\}, G}(x_1) - y_{\{2\}, G}(x_2) \label{eq:y12}
\end{align}

Since the components form a coupled system where the components are defined in interdependence of each other, finding the solution is not straight forward, even for simple examples.

\subsubsection*{Hooker method}
An alternative way to phrase the problem can be found in \cite{hooker2007}.
To find the generalized fANOVA components, we can formulate a minimization problem for each of them. 
\begin{align}
y_{\emptyset} &= \arg\min_{c \in \mathbb{R}} \int_{\mathbb{R}^2} 
\left( g(x_1, x_2) 
- \left( c + y_{\{1\}}(x_1) + y_{\{2\}}(x_2) + y_{\{1,2\}}(x_1, x_2) \right) \right)^2 \nonumber \\
&\qquad\cdot f(x_1, x_2)\, dx_1 dx_2 \label{eq:y0_hooker} \\
y_{\{1\}}(x_1) &= \arg\min_{h_1 \in L^2(\mathbb{R})} \int_{\mathbb{R}^2} 
\left( g(x_1, x_2) 
- \left( y_{\emptyset} + h_1(x_1) + y_{\{2\}}(x_2) + y_{\{1,2\}}(x_1, x_2) \right) \right)^2 \nonumber \\
&\qquad\cdot f(x_1, x_2)\, dx_1 dx_2 \label{eq:y1_hooker} \\
y_{\{2\}}(x_2) &= \arg\min_{h_2 \in L^2(\mathbb{R})} \int_{\mathbb{R}^2} 
\left( g(x_1, x_2) 
- \left( y_{\emptyset} + y_{\{1\}}(x_1) + h_2(x_2) + y_{\{1,2\}}(x_1, x_2) \right) \right)^2 \nonumber \\
&\qquad\cdot f(x_1, x_2)\, dx_1 dx_2 \label{eq:y2_hooker} \\
y_{\{1,2\}}(x_1, x_2) &= \arg\min_{h_{12} \in L^2(\mathbb{R}^2)} \int_{\mathbb{R}^2} 
\left( g(x_1, x_2) 
- \left( y_{\emptyset} + y_{\{1\}}(x_1) + y_{\{2\}}(x_2) + h_{12}(x_1, x_2) \right) \right)^2 \nonumber \\
&\qquad\cdot f(x_1, x_2)\, dx_1 dx_2 \label{eq:y12_hooker}
\end{align}


The least-squares problems are solved subject to the following constraints, which ensure that the resulting components are zero centred and hierarchically orthogonal:
\begin{align*}
\int_{\mathbb{R}^2} y_{\{1\}}(x_1) \cdot f(x_1, x_2)\, dx_1 dx_2 &= 0 \\[1ex]
\int_{\mathbb{R}^2} y_{\{2\}}(x_2) \cdot f(x_1, x_2)\, dx_1 dx_2 &= 0 \\[1.4ex]
\int_{\mathbb{R}} y_{\{1,2\}}(x_1, x_2) \cdot f(x_1, x_2)\, dx_1 &= 0 \quad \forall x_2 \\[1ex]
\int_{\mathbb{R}} y_{\{1,2\}}(x_1, x_2) \cdot f(x_1, x_2)\, dx_2 &= 0 \quad \forall x_1
\end{align*}


Conceptually \cite{hooker2007} is doing nothing other than a projection. Earlier, we established that a projection is the same as the conditional expected value, and fANOVA can be expressed via the conditional expected value. This means from the initial idea, we do not change anything apart from the fact that we have to integrate via the joint pdf, but this is something one is ``forced'' to under dependence, not something one ``invents''. However, projections onto subspaces become more difficult under dependence; therefore, setting these constraints explicitly is necessary to ensure (hierarchical) orthogonality.\par

Obtaining an analytical solution for either of the methods is tedious even for our simple example. We leave it at the problem formulation, so that we have the comparison between which problem one has to solve the classical case versus the generalized case. In the next section, we sketch ways to estimate the fANOVA components conceptually.

% Now a correction of the example from the beginning would need to follow...
% --> If not what we thought, what are the real generalized fANOVA components then??
% but maybe i have to come to the (very unsatifying) end that i cannot correct the above components 
% because solving this coupled system is too complex...?

% estimating them is more involved, we need to introduce this in the next section
% more or less formal intro to Hookers general estimation approach



