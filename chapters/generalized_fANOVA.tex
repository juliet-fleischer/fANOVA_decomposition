% setup
% space we live in, functions we deal with
The chapter is based on \cite{hooker2007}. We want to let go of two key assumptions of the classical fANOVA decomposition (as introduced by \cite{sobol1993sensitivity}): We widen the input domain to the reel number line, i.e. we now allow for the measure space $(X, \mathcal{F}, \nu) = (\mathbb{R}^n, \mathcal{B}(\mathbb{R}^n), dw(x))$. This goes hand in hand with dropping the assumption about the uniform distribution of the $X_i$. Further, we investigate what happens when the variables are no longer independent of each other.\par
The inner product in $\mathcal{L}^2(\mathbb{R}^n, \mathcal{B}(\mathbb{R}^n), dw(x))$ is now defined more generally as the integral of a weighted product:
\[

\]

The norm is given by 
% definition 
The general definition of the function $f(x)$ as a weighted sum stays the same (\autoref{eq:fanova_decomposition}). What changes is the definition of the fANOVA components.

% zero-mean constraint

% orthogonality

% Comparison table

% Comparison Example