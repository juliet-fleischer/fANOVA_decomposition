% setup
% space we live in, functions we deal with
Again multiple work on generalization of fANOVA, e.g. \cite{hooker2007,rahman2014, chastaing2012}.
We want to let go of two key assumptions of the classical fANOVA decomposition (as introduced by \cite{sobol1993sensitivity}): We widen the input domain to the multidimensional reel number line, i.e. we now work in the measure space $(X, \mathcal{F}, \nu) = (\mathbb{R}^n, \mathcal{B}(\mathbb{R}^n), dw(x))$. This goes hand in hand with dropping the assumption about the uniform distribution of the $X_i$. Further, we investigate what happens when the variables are no longer independent of each other.\par
The inner product on $\mathcal{L}^2(\mathbb{R}^n, \mathcal{B}(\mathbb{R}^n), dw(x))$ is now defined more generally as the integral of a weighted product:
\[
\langle f, g \rangle = \int f(x) g(x) \, d\nu(x) \quad \forall f, g \in \mathcal{L}^2 \quad \text{with} \quad \nu(dx) = w(x)dx
\]

The norm is given by 
\[
\|f\|_{w} = \sqrt{\langle f, f \rangle_{w}} = \sqrt{\int f^2(x) \, w(x)dx} \quad \forall f \in \mathcal{L}^2
\]
% definition 
The general definition of the function $f(x)$ as a weighted sum stays the same (see \autoref{eq:fanova_decomposition}). What changes is the definition of the fANOVA components. The components are simultaneously defined as:
\begin{equation}
\left\{ f_u(x_u) \,\middle|\, u \subseteq d \right\}
= \arg\min_{\{g_u \in L^2(\mathbb{R}^u)\}_{u \subseteq d}} 
\int \left( \sum_{u \subseteq d} g_u(x_u) - f(x) \right)^2 w(x) \, dx
\label{eq:fanova_decomposition_generalized}
\end{equation}
There is a key difference to the classical definition: All the components are defined simultaneously via the orthogonal projections of the original function $f(x)$. This means the components $f_u$ are a set of functions that jointly minimize the weighted squared difference to the original function $f(x)$ and fulfil the generalized zero-mean constraint and hierarchical orthogonality (both defined in the following). A natural choice for the weights $w(x)$ is the probability distribution of the $x_i$ \citep{hooker2007}.\par

We require the fANOVA terms to be centred around the grand mean, in the same way as we did for the classical approach.
\cite{hooker2007} formulates this in a generalized zero-mean condition for dependent variables:
\begin{equation}
\forall u \subseteq d,\ \forall i \in u: \quad 
\int f_u(x_u)\, w(x)\, dx_i\, dx_{-u} = 0
\label{eq:zero_mean_condition_generalized}
\end{equation}
% {\color{blue}What is the conceptual difference here? How can I explain the difference in word?}

% orthogonality
Orthogonality of the fANOVA terms plays and important role. It ensures that they represent isolated effects which makes the interpretation of fANOVA so useful in practice.
In contrast to the classical fANOVA, we set a hierarchical orthogonality constraint (instead of a general orthogonality constraint):
\begin{equation}
\forall v \subseteq u,\ \forall g:\quad 
\int f_u(x_u)\ g_v(x_v) w(x)\ dx = 0
\label{eq:orthogonality_generalized}
\end{equation}
{\color{blue} I am always puzzled by this definition because $v$ could theoretically be equal to $u$ which would require the function to be orthogonal to itself. But wanting this for all functions g somehow changes something, but I am not super clear why. Would it be correct to write:}
\begin{equation}
\forall v \subset u :\quad 
\int f_u(x_u)\ g_v(x_v) w(x)\ dx = 0
\label{eq:orthogonality_generalized}
\end{equation}


\begin{table}[htbp]
\centering
\begin{adjustbox}{max width=\textwidth}
\renewcommand{\arraystretch}{3.0}
\begin{tabular}{|c|c|c|} % |p{3.2cm}|p{6.5cm}|p{6.5cm}|
\hline
\textbf{Category}  & \textbf{Classical} & \textbf{Generalized} \\
\hline
Measure space & $([0,1]^n, \mathcal{B}([0,1]^n))$ & $(\mathbb{R}^n, \mathcal{B}(\mathbb{R}^n))$ \\
\hline
Measure &
$\mathbb{P}: \mathcal{B}([0,1]^n) \rightarrow [0,1]$ &
$\mu: \mathcal{B}(\mathbb{R}^n) \rightarrow [0,\infty)$,
where $\mu(A) = \int_A w(x)\,dx$, $w(x) = \frac{d\mu}{d\lambda}$ \\
\hline
Distribution assumption &
$\mathbf{X} = (X_1, \dots, X_n) \overset{\text{iid}}{\sim} \mathcal{U}([0,1])$ &
$\mathbf{X} = (X_1, \dots, X_n) \sim \text{any distribution}$ \\
\hline
Random Variable &
$\mathbf{X}: \Omega \rightarrow [0,1]^n$, $\mu := \mathbf{X}_\# \mathbb{P}$ &
$\mathbf{X}_*: \Omega \rightarrow \mathbb{R}^n$, $w(x)\,dx = \mathbf{X}_\# \mathbb{P}$ \\
\hline
Inner product &
$\langle f, g \rangle = \int f(x)g(x)\,dx$ &
$\langle f, g \rangle_w = \int f(x)g(x)\,w(x)\,dx$ \\
\hline
Norm &
$\|f\| = \left( \int f(x)^2\,dx \right)^{1/2} = \sqrt{\mathbb{E}[f(\mathbf{X})^2]}$ &
$\|f\|_w = \left( \int f(x)^2 w(x)\,dx \right)^{1/2} = \sqrt{\mathbb{E}[f(\mathbf{X})^2]}$ \\
\hline
fANOVA components &
$f_u(x) = \int_{x_{-u}} \left( F(x) - \sum_{v \subset u} f_v(x) \right) dx_{-u}$ &
$\{f_u(x_u)\}_{u \subset d} = \arg\min_{\{g_u \in L^2(\mathbb{R}^u)\}} \int \left( \sum_{u \subset d} g_u(x_u) - F(x) \right)^2 w(x)\,dx$ \\
\hline
Zero-mean constraint &
$\int f_u(x_u)\,dx_u = 0$ for $u \ne \emptyset$ &
$\forall u \subset d, \forall i \in u:\ \int f_u(x_u)\,w(x)\,dx_i\,dx_{-u} = 0$ \\
\hline
Orthogonality &
$\int f_u(x_u) f_v(x_v)\,dx = 0$ for $u \ne v$ &
$\forall v \subset u,\ \forall g_v:\ \int f_u(x_u) g_v(x_v) w(x)\,dx = 0$ \\
\hline
\end{tabular}
\end{adjustbox}
\caption{Comparison of classical and generalized functional ANOVA (fANOVA) decompositions.}
\label{tab:fanova_comparison}
\end{table}

\subsubsection*{Example 1}

\subsubsection*{Example 2: Multivariate Normal Inputs}

We return to our example function \(g(x_1, x_2) = a + x_1 + 2x_2 + x_1 x_2\) and assume that:

\[
\begin{pmatrix}
X_1 \\
X_2
\end{pmatrix}
\sim \mathcal{N}\left(
\begin{pmatrix} 0 \\ 0 \end{pmatrix},
\begin{pmatrix}
1 & \rho \\
\rho & 1
\end{pmatrix}
\right), \quad \rho > 0
\]

From the properties of the MVN, we know that marginal distributions are standard normal:
\[
X_i \sim \mathcal{N}(0, 1) \quad \text{for } i = 1, 2
\]

We also know that the conditional distributions are:
\[
X_1 \mid X_2 = x_2 \sim \mathcal{N}(\rho x_2, 1 - \rho^2), \quad
X_2 \mid X_1 = x_1 \sim \mathcal{N}(\rho x_1, 1 - \rho^2)
\]

The constant term \( f_0 \) is given by:
\begin{align*}
f_0 &= \mathbb{E}[g(X_1, X_2)] 
= a + \mathbb{E}[X_1] + 2\mathbb{E}[X_2] + \mathbb{E}[X_1 X_2] \\
&= a + \mathbb{E}[X_1 X_2] 
= a + \left( \text{Cov}(X_1, X_2) + \mathbb{E}[X_1]\mathbb{E}[X_2] \right) \\
&= a + \rho
\end{align*}

The main effects can be computed as:
\begin{align*}
f_1(x_1) 
&= \mathbb{E}[g(X_1, X_2) | X_1 = x_1] - f_0 \\
&= \mathbb{E}[a + x_1 + 2X_2 + x_1 X_2 | X_1 = x1] - (a + \rho) \\
&= a + x_1 + 2\mathbb{E}[X_2 | X_1 = x_1] + x_1 \mathbb{E}[X_2 | X_1 = x_1] - a - \rho \\
&= x_1 - \rho + \rho(2 + x_1) \\
f_2(x_2) 
&= \mathbb{E}[g(X_1, X_2) \mid X_2 = x_2] - f_0 \\
&= \mathbb{E}[a + X_1 + 2x_2 + X_1 x_2 \mid X_2 = x_2] - (a + \rho) \\
&= a + 2x_2 + x_2 \mathbb{E}[X_1 \mid X_2 = x_2] - a - \rho \\
&= 2x_2 + x_2^2\rho + \rho
\end{align*}

Finally, the interaction effect \( f_{12}(x_1, x_2) \) is given by:
\begin{align*}
f_{12}(x_1, x_2) 
&= g(x_1, x_2) - f_0 - f_1(x_1) - f_2(x_2) \\
&= a + x_1 + 2x_2 + x_1 x_2 - (a + \rho) - (x_1 - \rho + 2\rho x_1 + \rho x_1^2) - (2x_2 + x_2^2 \rho + \rho) \\
&= x_1 x_2 - 2\rho x_1 - \rho x_1^2 - \rho x_2^2 - \rho
\end{align*}
