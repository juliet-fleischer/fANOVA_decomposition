The literature around fANOVA can be grouped into several thematic clusters. Each highlights a different angle on why fANOVA has proven useful and points to why a unified presentation is needed.\par

The underlying principle of the hierarchical, additive decomposition of a function dates back to \cite{hoeffding1948}. In his seminal work on U-statistics, he introduced the Hoeffding decomposition.
Though originally framed around estimators, this decomposition laid the groundwork for fANOVA by showing how a symmetric function can be written as a sum of mutually orthogonal component functions of increasing dimensionality.\par
Independently, \cite{sobol1993sensitivity} proved that any square integrable function on the unit hypercube can be decomposed into a sum of mutually orthogonal and zero-centered component functions.
% He originally called it ``decomposition into summands of different dimension'' and later renames it ``ANOVA-representation'' \citep{sobol2001}; now referred to as the fANOVA decomposition.
The foundational work on fANOVA shows, that it is rooted in rigorous mathematical theory, and provides a principled way to break down complex multivariate functions into interpretable, orthogonal parts.\par

A second strand of work explores how fANOVA underlies non‑parametric modeling.
\citet{takemura1983} introduced tensor‑analysis of ANOVA decompositions, laying the theoretical foundation. \citet{stone1994} applied fANOVA ideas to polynomial splines and generalized additive models. \citet{gu2013} extended this into smoothing‑spline ANOVA frameworks for flexible regression estimation. Their work demonstrates, that fANOVA not only provides a theoretical decomposition, but also serves as a basis for widely-used non‑parametric statistical models featuring additive structure and controlled interactions.\par

Perhaps the most well-known application of fANOVA is in variance‑based sensitivity analysis. Sobol’s original decomposition led directly to a variance decomposition, on which Sobol' indices are based.
Work from \cite{owen2013, owen2014} modernized this framework, introducing efficient estimation strategies and generalized indices suited to quasi‑Monte Carlo methods. \citet{borgonovo2022} further advanced the field with mixture‑based generalizations of fANOVA for uncertainty quantification.\par

Classical fANOVA requires independent input variables, which is a strong assumption in many real‑world applications. Therefore, a stream of literature is concerned with the generalization of fANOVA to dependent variables.
While \cite{hooker2007} was the first to present a generalized fANOVA framework, many other researchers were inspired by his work to create modifications of this \cite{rahman2014,chastaing2012,ilidrissi2025}.
We see the generalization as central part of the basis of the fANOVA decomposition and therefore will also present it in this thesis.\par

A recent cluster of literature studies fANOVA for model interpretability. There is work of \cite{lengerich2020, konig2024, choi2025} that all enhance interpretability by using fANOVA to identify and disentangle variable interactions.
Then there is work done in the explicit context of IML, where fANOVA can be used as a model-agnostic tool \citep{hooker2004,fumagalli2025} or as foundational principle to build inherently interpretable models \citep{hu2025}. fANOVA-based interpretability methods is probably the most novel field of fANOVA in which research is actively ongoing.\par

Finally, there are specific domains of statistics, such as geostatistics, where fANOVA-based Kriging models are designed \citep{muehlenstaedt2012} or
complex functions arising in computational finance are studied \citep{liu2006}.







