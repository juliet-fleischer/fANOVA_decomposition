\section*{Proof of classical fANOVA decomposition}
Here we show the adjusted proof of Theorem 1 in \cite{sobol1993sensitivity}.
\begin{theorem}
    Any function $y$, which is integrable over the unit hypercube $[0, 1]^k$, has a unique fANOVA expansion of the form:
    \begin{align*}
        y(\boldsymbol{X}) = \sum_{u \subseteq \{1, \dots, N\}} y_{u}(\boldsymbol{X}_u),
    \end{align*}
    subject to the constraint that \autoref{prop:zero_mean_classical} is satisfied.
\end{theorem}
Sobol' proofs existence and uniqueness of the fANOVA decomposition by showing how the summands of the desired decomposition look and constructing them in such a way that they have the zero-mean property.

\begin{proof}
    Assume that $\boldsymbol{X}$ is an $N$-dimensional vector of independent random variables and
    that the still unspecified fANOVA components have zero-mean. He defines the integral w.r.t. all variables except for the ones with indices in \( v \):
    \begin{align*}
        g_v(\boldsymbol{x}_v) &= \int_{[0, 1]^{N - |v|}} y(\boldsymbol{x}) f_{\boldsymbol{X}}(\boldsymbol{x}) \, d\nu (\boldsymbol{x}_{-v})
    \end{align*}

He then builds the fANOVA terms subsequently and shows that they indeed satisfy the desired properties.\par
The very first term in the decomposition is the integral of \( y \) with respect to all variables:

\begin{align*}
    y_{\emptyset} = \int_{\mathbb{R}^N} y(\boldsymbol{x}) f_{\boldsymbol{X}}(\boldsymbol{x}) \, d\nu(\boldsymbol{x})
\end{align*}

This integral exists because \( y \in \mathcal{L}^2(\mathbb{R}^N, f_{\boldsymbol{X}}(\boldsymbol{x}) d\nu) \), and the product measure is finite on the domain.

Next, Sobol' derives the one-dimensional fANOVA terms. For this, he takes the integral of \autoref{def:fanova_decomposition} w.r.t. all variables except for the one with index \( i \), so $v_1 = \{i\}$:
\begin{align*}
    \int_{\mathbb{R}^{N-1}} y(\boldsymbol{x}) f_{\boldsymbol{X}}(\boldsymbol{x}) d\nu (\boldsymbol{x}_{-v_1}) 
    &= \int_{\mathbb{R}^{N-1}} \sum_{u \subseteq \{1, \dots, N\}} y_{u}(\boldsymbol{x}_u) f_{\boldsymbol{X}}(\boldsymbol{x}) d\nu (\boldsymbol{x}_{-v_1}) \\
    &= \sum_{u \subseteq \{1, \dots, N\}} \int_{\mathbb{R}^{N-1}} y_{u}(\boldsymbol{x}_u) f_{\boldsymbol{X}}(\boldsymbol{x}) d\nu (\boldsymbol{x}_{-v_1}) \\
    &= \sum_{u \subseteq \{1, \dots, N\}} \int_{\mathbb{R}^{N-1}} y_{u}(\boldsymbol{x}_u) \left( \prod_{j=1}^N f_{{\{j\}}}(x_j) \right) d\nu (\boldsymbol{x}_{-v_1})
\end{align*}

For every summand \( y_u(\boldsymbol{x}_u) \) with \( u \not\ni i \), the integrand does not depend on \( x_i \), and thus vanished due to the zero-mean constraint.
Similarly, for any term \( y_u(\boldsymbol{x}_u) \) with \( i \in u \) and \( |u| > 1 \), the integration will include at least one other variable in \( u \), again causing the integral to vanish.
In the end, only the constant term \( y_{\emptyset} \) and the one-dimensional term \( y_{\{i\}}(x_i) \) remain, which depend only on \( x_i \) and are not integrated.
Therefore, we can derive the simplified expression:
\begin{align*}
    \int_{\mathbb{R}^{N-1}} y(\boldsymbol{x}) f_{\boldsymbol{X}}(\boldsymbol{x}) d\nu (\boldsymbol{x}_{-v_1}) = y_{\emptyset} + y_{\{i\}}(x_i).
\end{align*}

This equation allows to define the one-dimensional term \( y_{\{i\}} \) explicitly as:
\begin{align*}
    y_{\{i\}}(x_i) = \int_{\mathbb{R}^{N-1}} y(\boldsymbol{x}) f_{\boldsymbol{X}}(\boldsymbol{x}) d\nu (\boldsymbol{x}_{-v_1}) - y_{\emptyset}.
\end{align*}

Next, he considers $v_2 = \{i, j\}$. The ANOVA decomposition is integrated over all variables except $x_i$ and $x_j$:

\begin{align*}
    \int_{\mathbb{R}^{N-2}} y(\boldsymbol{x}) f_{\boldsymbol{X}}(\boldsymbol{x}) d\nu(\boldsymbol{x}_{-\{i,j\}}) 
    &= \int_{\mathbb{R}^{N-2}} \sum_{u \subseteq \{1, \dots, N\}} y_u(\boldsymbol{x}_u) f_{\boldsymbol{X}}(\boldsymbol{x}) d\nu(\boldsymbol{x}_{-\{i,j\}}) \\
    &= \sum_{u \subseteq \{1, \dots, N\}} \int_{\mathbb{R}^{N-2}} y_u(\boldsymbol{x}_u) \left( \prod_{j=1}^N f_{X_j}(x_j) \right) d\nu(\boldsymbol{x}_{-\{i,j\}}) \\
    &= y_{\emptyset} + y_{\{i\}}(x_i) + y_{\{j\}}(x_j) + y_{\{i,j\}}(x_i, x_j)
\end{align*}

Hence, the two-dimensional components are given by:

\begin{align*}
    y_{\{i,j\}}(x_i, x_j) 
    = \int_{\mathbb{R}^{N-2}} y(\boldsymbol{x}) f_{\boldsymbol{X}}(\boldsymbol{x}) d\nu(\boldsymbol{x}_{-\{i,j\}})
    - y_{\emptyset} - y_{\{i\}}(x_i) - y_{\{j\}}(x_j)
\end{align*}
One can continue this process for all combinations of indices \( v \subseteq \{1, \ldots, N\} \) to derive the corresponding fANOVA terms \( y_v(\boldsymbol{x}_v) \).\par
Now let \( v \subseteq \{1, \ldots, N\} \). The general expression for the component \( y_v(\boldsymbol{x}_v) \) is given by:

\begin{align*}
    y_v(\boldsymbol{x}_v) = \int_{\mathbb{R}^{N - |v|}} y(\boldsymbol{x}) f_{\boldsymbol{X}}(\boldsymbol{x}) d\nu(\boldsymbol{x}_{-v}) 
    - \sum_{u \subsetneq v} y_u(\boldsymbol{x}_u)
\end{align*}

The last term is the decomposition integrated with respect to no variables, i.e., the function itself:
\begin{align*}
    y_{\{1, \ldots, N\}}(\boldsymbol{x}) = y(\boldsymbol{x}) 
    - \sum_{u \subsetneq \{1, \ldots, N\}} y_u(\boldsymbol{x}_u)
\end{align*}

Finally, it remains to verify that the constructed component functions satisfy the zero-mean constraint. Let \( v \subseteq \{1, \ldots, N\} \), and let \( i \in v \). Then:

\begin{align*}
    \int y_v(\boldsymbol{x}_v) f_{X_i}(x_i) \, d\nu(x_i)
    &= \int \left( \int y(\boldsymbol{x}) f_{\boldsymbol{X}}(\boldsymbol{x}) \, d\nu(\boldsymbol{x}_{-v}) - \sum_{u \subsetneq v} y_u(\boldsymbol{x}_u) \right) f_{\{i\}}(x_i) \, d\nu(x_i) \\
    &= \int \left( \int y(\boldsymbol{x}) f_{\boldsymbol{X}}(\boldsymbol{x}) \, d\nu(\boldsymbol{x}_{-v}) \right) f_{\{i\}}(x_i) \, d\nu(x_i) 
    - \sum_{u \subsetneq v} \int y_u(\boldsymbol{x}_u) f_{\{i\}}(x_i) \, d\nu(x_i) \\
    &= \int y(\boldsymbol{x}) f_{\boldsymbol{X}}(\boldsymbol{x}) \, d\nu(\boldsymbol{x}_{-v}) d\nu(x_i)
    - \sum_{u \subsetneq v} \int y_u(\boldsymbol{x}_u) f_{\boldsymbol{X}_u}(\boldsymbol{x}_u) \, d\nu(\boldsymbol{x}_u)
\end{align*}

The first term integrates out all of \( \boldsymbol{x}_v \), leaving \( y_{\emptyset} \). Each term in the sum vanishes by the zero-mean property of lower-order components:

\begin{align*}
    \int y_v(\boldsymbol{x}_v) f_{\{i\}}(x_i) \, d\nu(x_i) = y_{\emptyset} - y_{\emptyset} = 0
\end{align*}

Thus, every component \( y_v(\boldsymbol{x}_v) \) satisfies:

\begin{align*}
    \int y_v(\boldsymbol{x}_v) f_{\{i\}}(x_i) \, d\nu(x_i) = 0, \quad \text{for all } i \in v.
\end{align*}
\end{proof}

