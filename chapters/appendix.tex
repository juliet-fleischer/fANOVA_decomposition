\section*{Proof of classical fANOVA decomposition}
Here we show the proof of Theorem 1 in \cite{sobol1993sensitivity}.
\begin{theorem}
    Any function $y$, which is integrable over the unit hypercube $[0, 1]^k$, has a unique fANOVA expansion of the form:
    \begin{align}
        y(\boldsymbol{X}) = \sum_{u \subseteq \{1, \dots, N\}} y_{u}(\boldsymbol{X}_u),
    \end{align}
    subject to the zero mean constraint \autoref{eq:zero_mean_c}.
\end{theorem}
Sobol proofs existence and uniqueness of the fANOVA decomposition by showing how the summands of the desired decomposition look and showing that they satisfy the desired zero mean property.

\begin{proof}
    Assume that $\boldsymbol{X}$ is an $N$-dimensional vector of independent random variables and that the zero mean property holds for the - still abstract - fANOVA terms. We define the integral w.r.t. all variables except for the ones with indices in \( v \):
    \begin{align}
        g_v(\boldsymbol{x}_v) &= \int_{[0, 1]^{N - |v|}} y(\boldsymbol{x}) f_{\boldsymbol{X}}(\boldsymbol{x}) \, d\nu (\boldsymbol{x}_{-v})
    \end{align}

The very first term in the decomposition is the integral of \( y \) with respect to all variables:

\begin{align}
    y_{\emptyset} = \int_{\mathbb{R}^N} y(\boldsymbol{x}) f_{\boldsymbol{X}}(\boldsymbol{x}) \, d\nu(\boldsymbol{x}) \label{eq:y_emptyset}
\end{align}

This integral exists because \( y \in L^2(\mathbb{R}^N, f_{\boldsymbol{X}} d\nu) \), and the product measure is finite on the domain.

Next, Sobol derives the one dimensional fANOVA terms. For this, take the integral of \autoref{eq:fanova_decomposition} w.r.t. all variables except for the one with index \( i \), so $v_1 = \{i\}$:
\begin{align}
    \int_{\mathbb{R}^{N-1}} y(\boldsymbol{x}) f_{\boldsymbol{X}}(\boldsymbol{x}) d\nu (\boldsymbol{x}_{-v_1}) 
    &= \int_{\mathbb{R}^{N-1}} \sum_{u \subseteq \{1, \dots, N\}} y_{u}(\boldsymbol{x}_u) f_{\boldsymbol{X}}(\boldsymbol{x}) d\nu (\boldsymbol{x}_{-v_1}) \\
    &= \sum_{u \subseteq \{1, \dots, N\}} \int_{\mathbb{R}^{N-1}} y_{u}(\boldsymbol{x}_u) f_{\boldsymbol{X}}(\boldsymbol{x}) d\nu (\boldsymbol{x}_{-v_1}) \\
    &= \sum_{u \subseteq \{1, \dots, N\}} \int_{\mathbb{R}^{N-1}} y_{u}(\boldsymbol{x}_u) \left( \prod_{j=1}^N f_{X_j}(x_j) \right) d\nu (\boldsymbol{x}_{-v_1})
\end{align}

For every summand \( y_u(\boldsymbol{x}_u) \) with \( u \not\ni i \), the integrand does not depend on \( x_i \), and thus vanished due to the zero mean constraint.
Similarly, for any term \( y_u(\boldsymbol{x}_u) \) with \( i \in u \) and \( |u| > 1 \), the integration will include at least one other variable in \( u \), again causing the integral to vanish.
In the end, only the constant term \( y_{\emptyset} \) and the one-dimensional term \( y_{\{i\}}(x_i) \) remain, which depend only on \( x_i \) and are not integrated.
Therefore, we can derive the simplified expression:
\begin{align}
    \int_{\mathbb{R}^{N-1}} y(\boldsymbol{x}) f_{\boldsymbol{X}}(\boldsymbol{x}) d\nu (\boldsymbol{x}_{-v_1}) = y_{\emptyset} + y_{\{i\}}(x_i).
\end{align}

This equation allows us to define the one-dimensional term \( y_{\{i\}} \) explicitly as:
\begin{align}
    y_{\{i\}}(x_i) = \int_{\mathbb{R}^{N-1}} y(\boldsymbol{x}) f_{\boldsymbol{X}}(\boldsymbol{x}) d\nu (\boldsymbol{x}_{-v_1}) - y_{\emptyset}. \label{eq:y_i}
\end{align}

Now consider $v_2 = \{1, 2\}$. We integrate the ANOVA decomposition over all variables except $x_1$ and $x_2$:

\begin{align}
    \int_{\mathbb{R}^{N-2}} y(\boldsymbol{x}) f_{\boldsymbol{X}}(\boldsymbol{x}) d\nu(\boldsymbol{x}_{-\{1,2\}}) 
    &= \int_{\mathbb{R}^{N-2}} \sum_{u \subseteq \{1, \dots, N\}} y_u(\boldsymbol{x}_u) f_{\boldsymbol{X}}(\boldsymbol{x}) d\nu(\boldsymbol{x}_{-\{1,2\}}) \\
    &= \sum_{u \subseteq \{1, \dots, N\}} \int_{\mathbb{R}^{N-2}} y_u(\boldsymbol{x}_u) \left( \prod_{j=1}^N f_{X_j}(x_j) \right) d\nu(\boldsymbol{x}_{-\{1,2\}}) \\
    &= y_{\emptyset} + y_{\{1\}}(x_1) + y_{\{2\}}(x_2) + y_{\{1,2\}}(x_1, x_2)
\end{align}

Hence, the two-dimensional component is given by:

\begin{align}
    y_{\{1,2\}}(x_1, x_2) 
    = \int_{\mathbb{R}^{N-2}} y(\boldsymbol{x}) f_{\boldsymbol{X}}(\boldsymbol{x}) d\nu(\boldsymbol{x}_{-\{1,2\}})
    - y_{\emptyset} - y_{\{1\}}(x_1) - y_{\{2\}}(x_2) \label{eq:y_12}
\end{align}
We can continue this process for all combinations of indices \( v \subseteq \{1, \ldots, N\} \) to derive the corresponding fANOVA terms \( y_v(\boldsymbol{x}_v) \).
Now let \( v \subseteq \{1, \ldots, N\} \). The general expression for the component \( y_v(\boldsymbol{x}_v) \) is given by:

\begin{align}
    y_v(\boldsymbol{x}_v) = \int_{\mathbb{R}^{N - |v|}} y(\boldsymbol{x}) f_{\boldsymbol{X}}(\boldsymbol{x}) d\nu(\boldsymbol{x}_{-v}) 
    - \sum_{u \subsetneq v} y_u(\boldsymbol{x}_u) \label{eq:y_v_general}
\end{align}

The last term is the decomposition integrated with respect to no variables, i.e., the function itself:
\begin{align}
    y_{\{1, \ldots, N\}}(\boldsymbol{x}) = y(\boldsymbol{x}) 
    - \sum_{u \subsetneq \{1, \ldots, N\}} y_u(\boldsymbol{x}_u) \label{eq:y_full}
\end{align}

Finally, we verify that the constructed component functions satisfy the zero mean constraint. Let \( v \subseteq \{1, \ldots, N\} \), and let \( i \in v \). Then:

\begin{align}
    \int y_v(\boldsymbol{x}_v) f_{X_i}(x_i) \, d\nu(x_i)
    &= \int \left( \int y(\boldsymbol{x}) f_{\boldsymbol{X}}(\boldsymbol{x}) \, d\nu(\boldsymbol{x}_{-v}) - \sum_{u \subsetneq v} y_u(\boldsymbol{x}_u) \right) f_{X_i}(x_i) \, d\nu(x_i) \\
    &= \int \left( \int y(\boldsymbol{x}) f_{\boldsymbol{X}}(\boldsymbol{x}) \, d\nu(\boldsymbol{x}_{-v}) \right) f_{X_i}(x_i) \, d\nu(x_i) 
    - \sum_{u \subsetneq v} \int y_u(\boldsymbol{x}_u) f_{X_i}(x_i) \, d\nu(x_i) \\
    &= \int y(\boldsymbol{x}) f_{\boldsymbol{X}}(\boldsymbol{x}) \, d\nu(\boldsymbol{x}_{-v}) d\nu(x_i)
    - \sum_{u \subsetneq v} \int y_u(\boldsymbol{x}_u) f_{\boldsymbol{X}_u}(\boldsymbol{x}_u) \, d\nu(\boldsymbol{x}_u)
\end{align}

The first term integrates out all of \( \boldsymbol{x}_v \), leaving \( y_{\emptyset} \). Each term in the sum vanishes by the zero-mean property of lower-order components:

\begin{align}
    \int y_v(\boldsymbol{x}_v) f_{X_i}(x_i) \, d\nu(x_i) = y_{\emptyset} - y_{\emptyset} = 0
\end{align}

Thus, every component \( y_v(\boldsymbol{x}_v) \) satisfies:

\begin{align}
    \int y_v(\boldsymbol{x}_v) f_{X_i}(x_i) \, d\nu(x_i) = 0, \quad \text{for all } i \in v. \label{eq:zero_mean_verified}
\end{align}
This completes the proof of the existence and uniqueness of the fANOVA decomposition, as well as the verification of the zero mean property for all components.
\end{proof}


\section*{Square Integrability of \( f_1(x_1) \)}

For now we want to show that the single fANOVA term \( f_1(x_1) \) is square integrable, given that the original function $f(x) \in \mathcal{L}^2$. We need to show that:
\[
\int |f_1(x_1)|^2 \, dx_1 < \infty
\]

The single fANOVA term is defined as:
\[
f_1(x_1) = \int f(x) \, dx_{-1} - f_0
\]

We take the squared norm, and integrate w.r.t. \( x_1 \) to use the Cauchy-Schwarz inequality:
\[
\int |f_1(x_1)|^2 \, dx_1 
= \int \left| \int f(x) \, dx_{-1} - f_0 \right|^2 dx_1
\]

\[
= \int | (\int f(x) \, dx_{-1})^2 
- 2 \int f(x) \, dx_{-1} f_0 
+ f_0^2 | dx_1
\]

Break this into three terms:
\begin{align*}
(1): &\quad \int \left| \int f(x) \, dx_{-1} \right|^2 dx_1 
\leq \int \left( \int 1^2 \, dx_{-1} \right) \left( \int |f(x)|^2 \, dx_{-1} \right) dx_1 
= \int |f(x)|^2 \, dx < \infty \\
\\
(2): &\quad 2 \int \left( \int f(x) \, dx_{-1} \right) f_0 \, dx_1 
= 2 f_0 \int \left( \int f(x) \, dx_{-1} \right) dx_1 
= 2 f_0^2 < \infty \\
\\
(3): &\quad \int f_0^2 \, dx_1 = f_0^2 < \infty
\end{align*}

Since each term (1)–(3) is finite, and \( \int |f_1(x_1)|^2 dx_1 \) is a linear combination of them: \(\int |f_1(x_1)|^2 dx_1 < \infty\)



% We want to show that the square integrability of the original function \( f \) implies the square integrability of the fANOVA terms \( f_0, f_1, \ldots, f_k \).
% For simplicity, we restrict ourselves to a fixed number of dimensions and therefore a fixed set of indices $i = 1, 2, 3, 4$.
% To start, we define the cumulative fANOVA decomposition as follows:
% \begin{align}
%     g_1(x) &= \int f(x) dx_{-1} = \int f(x) dx_2 dx_3 dx_4 \\
%     g_2(x) &= \int f(x) dx_{-2} = \int f(x) dx_1 dx_3 dx_4 \\
%     g_{1,2}(x) &= \int f(x) dx_{-1, -2} = \int f(x) dx_3 dx_4 \\
%     g_{1,2,3}(x) &= \int f(x) dx_{-1, -2, -3} = \int f(x) dx_4 \\
%     g_{1,2,3,4}(x) &= \int f(x) dx_{-1, -2, -3, -4} = \int f(x)
% \end{align}
% Further, recall that the Cauchy-Schwarz inequality for two function $f, g$ is given by:
% \begin{align}
%     \left( \int f(x) g(x) dx \right)^2 \leq \left( \int f(x)^2 dx \right) \left( \int g(x)^2 dx \right)
% \end{align}

% Let $f(x) \in L^2(\mathbb{R}^4)$ with $x = (x_1, x_2, x_3, x_4)$
% We want to show that the term $g_{1,2}$ is square integrable, i.e. $\int |g_{1,2}(x)|^2 dx < \infty$. The reasoning for other terms will follow the same principle.\par
% To be able to use Schwarz inequality, we square the norm of $g_{1,2}(x)$:
% \begin{align}
%     |g_{1,2}(x_1, x_2)|^2 
%     &= \left( \int f(x_1, x_2, x_3, x_4) \, dx_3 dx_4 \right)^2 \\
%     &\leq ( \int 1^2 \, dx_3 dx_4) (\int |f(x_1, x_2, x_3, x_4)|^2 \, dx_3 dx_4) \\
%     &= \int |f(x_1, x_2, x_3, x_4)|^2 \, dx_3 dx_4
% \end{align}
% The statement is true for fixed values of $x_1$ and $x_2$. To show that it holds for all $x_1, x_2$, we integrate over $x_1$ and $x_2$:
% \begin{align}
%     \int |g_{1,2}(x_1, x_2)|^2 \, dx_1 dx_2 
%     &\leq \int \left( \int |f(x_1, x_2, x_3, x_4)|^2 \, dx_3 dx_4 \right) \, dx_1 dx_2 \\
%     &= \int |f(x_1, x_2, x_3, x_4)|^2 \, dx \leq \infty
% \end{align}
% We used Fubini's theorem to write the sequential integration as a single integral over the whole space, which is valid under the assumption that $f$ is square integrable.
% As a last step, we use the square integrability of the cumulative fANOVA terms to show square integrability of the single fANOVA terms. We give an example for $f_{1,2}$:
% \begin{align}
%     f_{1,2}(x) &= g_{1,2}(x) - f_0 - f_1(x_1) - f_2(x_2) \\
%     &= g_{1,2}(x) - \int f(x) dx - \int f(x)\, dx_{-1} - \int f(x) dx_{-2} \\
%     &= g_{1,2}(x) - \int f(x) dx - g_1(x) - g_2(x) \leq \infty
% \end{align}
% Since all the terms in the last row are square integrable, we deal with a linear combination of square integrable functions, which is also square integrable.
