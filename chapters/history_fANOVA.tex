% Good historical overview of fANOVA decomposition found in: Owen (2003), Takemura (1983)

\subsection{Early Work on fANOVA}
The main idea of the fANOVA decomposition is to decompose a statistical model into the sum of the main effects and interaction effects of its input variables. The underlying principle of fANOVA decomposition dates back to \cite{hoeffding1948}. In his famous paper he introduced U-statistics, along with the ``Hoeffding decomposition'', which allows to write a symmetric function of the data as a sum of orthogonal components. \cite{sobol1993sensitivity} used the same principle and applied it to deterministic mathematical models.
% Both methods involve the sum of orthogonal components and independent input variables.
He built on the originally called ``decomposition into summands of different dimension'' in \cite{sobol2001}, where he introduces Sobol indices and renames the method to the ``ANOVA-representation''. Sobol indices are now commonly used in sensitivity analysis. \cite{efron1981} use the idea of the decomposition to proof their famous lemma on jackknife variances. \cite{stone1994} mainly uses fANOVA decomposition to base smooth regression models with interactions on it and his paper is the building block for a broader body of work of fANOVA-based models {\color{blue}example citations needed}.


\subsection{Modern Work on fANOVA}
The fANOVA decomposition has a long history with roots in mathematical statistics and non-parametric estimation theory. In more recent years, the method has been rediscovered by the machine-learning community, especially in the context of interpretable machine learning (IML) and explainable AI (XAI). \cite{hooker2004} introduces the fANOVA decomposition with the goal of providing a global explanation method for black-box models. Since the assumptions of independent variables in classical fANOVA is often too restrictive in practice, \cite{hooker2007} generalizes the method to dependent variables. A recent paper by \cite{ilidrissi2025} can be seen as another approach to generalize the principle of fANOVA decomposition to dependent inputs.\par
There are specific domains of statistics, such as geostatistics, that explicitly build models on fANOVA framework (see \cite{muehlenstaedt2012} for fANOVA Kriging models). And recent work discovered interesting mathematical parallels between fANOVA and other IML methods, such as PDP \cite{friedman2001}, or Shapley values (\cite{fumagalli2025}, Herren, Owen preprint).

\cite{liu2006} use of fANOVA and sensitivity analysis for functions arising in computational finance.
\cite{owen2013} formal intro to fANOVA decomposition and generalization of Sobol indices.
Owen has generally a lot of work related to fANOVA decomposition, either lecture notes explaining the decomposition, methods based on it \cite{owen2003}, or deeper into sensitivity analysis and fANOVA \cite{owen2013}.

\textbf{fANOVA and U-statistics, fANOVA and sensitivity analysis, fANOVA and GAMs (with interactions)}

