% Good historical overview of fANOVA decomposition found in: Owen (2003), Takemura (1983)

\subsection{Early Work on fANOVA}
% Hoeffding decomposition as foundation
The main idea of the fANOVA decomposition is to decompose a statistical model into the sum of the main effects and interaction effects of its input variables. The underlying principle of fANOVA decomposition dates back to \cite{hoeffding1948}. In his seminal work(?) on estimators with asymptotical normal distribution, he introduced U-statistics, along with the ``Hoeffding decomposition'', which allows to write a symmetric function of the data as a sum of orthogonal components.
% Sobol: first explicity fANOVA decomposition + Sobol indices - Both methods involve the sum of orthogonal components and independent input variables.
\cite{sobol1993sensitivity} used the same principle and applied it to deterministic mathematical models. proofed existance of fANOVA decomposition for square integrable functions.
He built on the originally called ``decomposition into summands of different dimension'' in \cite{sobol2001}, where he introduces Sobol indices and renames the method to the ``ANOVA-representation''. For Sobol decomposing the function into the sum of fANOVA terms is actually not central, but what he is mostly interested is the variance decomposition which he shows follows from the fANOVA decomposition of a function.
This variance decomposition allows quantifying how much the variance of a single input variable contributes to the overall variance of the function. Thus, Sobol indices are commonly used in sensitivity analysis.
Sobol builds his main contributions around fANOVA on the 1) variance decomposition, but also proposes to use fANOVA for 2) variable selection/ dimensionality reduction (terms that contribute a lot to overall variance should be in the model).\par

% Other
\cite{efron1981} use the idea of the decomposition to proof their famous lemma on jackknife variances.

% Stone: GAM models based on fANOVA
% Stone builds on TAKEMURA, A. (1983). Tensor analysis of ANOVA decomposition. J. Amer. Statist. Assoc. 78 894900.
A true wave of fANOVA literature around the 1990s, where authors investigate fANOVA-based models, establish parallels to splines, study their theoretical properties (convergence, consistency, etc.), and prectical use cases (dimensionality reduction, etc.). All cited in \cite{huang1998a}.
\cite{stone1994} mainly uses fANOVA  decomposition to base smooth regression models with interactions on it and his paper is the building block for a broader body of work of fANOVA-based models (see for example \cite{Huang1996, huang1998a})
% so his work has to do with tensor product (which are basically interactions??), splines, additive models
Go deeper into some of the works? And a wrap up? Wrap up: fANOVA received a lot of attention in statistics literature, its mathematical properties were studied and was also used as intermediate step to proof or build other theories.

\subsection{Modern Work on fANOVA}
The fANOVA decomposition has a long history with roots in mathematical statistics and non-parametric estimation theory.

% Owens work on fANOVA decomposition (reinterpretation/ moderinization of classical fANOVA studies)
\cite{owen2013} formal intro to fANOVA decomposition and generalization of Sobol indices.
Owen has generally a lot of work related to fANOVA decomposition, either lecture notes explaining the decomposition, methods based on it \cite{owen2003}, or deeper into sensitivity analysis and fANOVA \cite{owen2013}.

% fANOVA and dependent input variables
Since the assumptions of independent variables in classical fANOVA is often too restrictive in practice, \cite{hooker2007} generalizes the method to dependent variables. A recent paper by \cite{ilidrissi2025} can be seen as another approach to generalize the principle of fANOVA decomposition to dependent inputs.\par

% fANOVA and IML
In more recent years, the method has been rediscovered by the machine-learning community, especially in the context of interpretable machine learning (IML) and explainable AI (XAI). \cite{hooker2004} introduces the fANOVA decomposition with the goal of providing a global explanation method for black-box models.
And recent work discovered interesting mathematical parallels between fANOVA and other IML methods, such as PDP \cite{friedman2001}, or Shapley values (\cite{fumagalli2025}, Herren, Owen preprint).

% fANOVA based ML models
GA2ML etc.

% fANOVA and specific domains
There are specific domains of statistics, such as geostatistics, that explicitly build models on fANOVA framework (see \cite{muehlenstaedt2012} for fANOVA Kriging models).
\cite{liu2006} use of fANOVA and sensitivity analysis for functions arising in computational finance.

fANOVA/ variance decomposition to reduce dimensionality, fANOVA to find additive structures (explanability and surrogate modelling), fANOVA for identifying interaction terms and variable dependencies.