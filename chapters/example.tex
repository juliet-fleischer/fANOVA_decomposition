\subsection{Multivariate Normal Inputs}

Consider the function \(g = a + X_1 + 2X_2 + X_1 X_2\). We assume that $\boldsymbol{X} = (X_1, X_2)^T$ follows a MVN distribution:

\[
\begin{pmatrix}
X_1 \\
X_2
\end{pmatrix}
\sim \mathcal{N}\left(
\begin{pmatrix} 0 \\ 0 \end{pmatrix},
\begin{pmatrix}
\sigma_1^2 & \rho_{12}\sigma_1 \sigma_2 \\
\rho_{12}\sigma_1 \sigma_2  & \sigma_2^2
\end{pmatrix}
\right)
\]
where we assume that $\sigma_1 = \sigma_2 = 1$, while we will vary the values for $\rho_{12}$ to distinguish between varying degrees of dependence between the two variables.
From the properties of the MVN, we know that marginal distributions are standard normal:
\[
X_i \sim \mathcal{N}(0, \sigma_i) \quad \text{for } i = 1, 2 \quad \text{and} \quad \sigma_i = 1
\]

We also know that the conditional distributions are:
\[
X_1 \mid X_2 = x_2 \sim \mathcal{N}(\rho_{12} x_2, 1 - \rho_{12}^2), \quad
X_2 \mid X_1 = x_1 \sim \mathcal{N}(\rho_{12} x_1, 1 - \rho_{12}^2)
\]

\subsubsection*{Case 1: Independent Inputs}
Computing the fANOVA decomposition of $g(x_1, x_2)$ by hand, we start with the constant term and make use of formulation via the expected value instead of the integral for notational simplicity:
\[
f_0 = \mathbb{E}[g_{1}(X_1, X_2)] = \mathbb{E}[a + X_1 + 2X_2 + X_1X_2] = \mathbb{E}[a] + \mathbb{E}[X_1] + 2\mathbb{E}[X_2] + \mathbb{E}[X_1X_2]
\]
Making use of the independence assumption of $x_1$ and $x_2$, the last term can be written as the product of the expected values. Additionally, given the zero-mean constraint, all terms, except for the constant, vanish and we obtain:
\[
f_0 = \mathbb{E}[a] + \mathbb{E}[X_1] + 2\mathbb{E}[X_2] + \mathbb{E}[X_1]\mathbb{E}[X_2] = a
\]
Under zero-mean constraint and independence, the main effects and the interaction effect can be computed as follows:
\begin{align*}
f_1(x_1) &= \mathbb{E}_{X_2}[g_{1}(x_1, X_2)] - f_0 \\
&= \mathbb{E}_{X_2}[a + x_1 + 2X_2 + x_1X_2] - a \\
&= x_1 + 2\mathbb{E}[X_2] + x_1\mathbb{E}[X_2] = x_1\\
f_2(x_2) &= \mathbb{E}_{X_1}[g_{1}(X_1, x_2)] - f_0 \\
&= \mathbb{E}_{X_1}[a + X_1 + 2x_2 + X_1x_2] - a \\
&= \mathbb{E}_{X_1}[X_1] + 2x_2 + x_2\mathbb{E}_{X_1}[X_1] = 2x_2\\
f_{12}(x_1, x_2) &= \mathbb{E}[g_{1}(x_1, x_2)] - f_0 - f_1(x_1) - f_2(x_2) \\
&= a + x_1 + 2x_2 + x_1x_2 - a - x_1 - 2x_2 = x_1x_2
\end{align*}

It comes as no surprise that in this simple case the fANOVA decomposition does not provide any additional insights, as the isolated effects can be directly seen from the function.
In case of independence the generalized fANOVA decomposition is of course equal to the classical one.
We show this simple example nevertheless to illustrate at which step which assumption is used. This motivates the need for a generalization of the fANOVA decomposition to dependent variables.

\subsubsection*{Case 2: Dependent Inputs (weak)}
Now assume $\rho_{12} = 0.2,\, \sigma_1 = \sigma_2 = 1$. The constant term \( f_0 \) is given by:
\begin{align*}
f_0 &= \mathbb{E}[g(X_1, X_2)] 
= a + \mathbb{E}[X_1] + 2\mathbb{E}[X_2] + \mathbb{E}[X_1 X_2] \\
&= a + \mathbb{E}[X_1 X_2] 
= a + \left( \text{Cov}(X_1, X_2) + \mathbb{E}[X_1]\mathbb{E}[X_2] \right) \\
&= a + \rho_{12}
\end{align*}

The main effects can be computed as:
\begin{align*}
f_1(x_1) 
&= \mathbb{E}[g(X_1, X_2) | X_1 = x_1] - f_0 \\
&= \mathbb{E}[a + x_1 + 2X_2 + x_1 X_2 | X_1 = x1] - (a + \rho_{12}) \\
&= a + x_1 + 2\mathbb{E}[X_2 | X_1 = x_1] + x_1 \mathbb{E}[X_2 | X_1 = x_1] - a - \rho_{12} \\
&= x_1 - \rho_{12} + \rho_{12}(2 + x_1) \\
f_2(x_2) 
&= \mathbb{E}[g(X_1, X_2) \mid X_2 = x_2] - f_0 \\
&= \mathbb{E}[a + X_1 + 2x_2 + X_1 x_2 \mid X_2 = x_2] - (a + \rho_{12}) \\
&= a + 2x_2 + x_2 \mathbb{E}[X_1 \mid X_2 = x_2] - a - \rho_{12} \\
&= 2x_2 + x_2^2\rho_{12} + \rho_{12}
\end{align*}

Finally, the interaction effect \( f_{12}(x_1, x_2) \) is given by:
\begin{align*}
f_{12}(x_1, x_2) 
&= g(x_1, x_2) - f_0 - f_1(x_1) - f_2(x_2) \\
&= a + x_1 + 2x_2 + x_1 x_2 - (a + \rho_{12}) - (x_1 - \rho_{12} + 2\rho_{12} x_1 + \rho_{12} x_1^2) - (2x_2 + x_2^2 \rho_{12} + \rho_{12}) \\
&= x_1 x_2 - 2\rho_{12} x_1 - \rho_{12} x_1^2 - \rho_{12} x_2^2 - \rho_{12}
\end{align*}

\subsection*{Case 3: Dependent Inputs (strong)}